In this chapter, we discuss how the theory behind quaternionic determinants - specifically, the Study Determinant - is built. With this, we extend theorem \ref{dnc} by showing that there are essentially no skew-coninvolutory quaternionic matrices of odd dimensions.  

\section{The Cayley Determinant and Aslaksen's Axioms} \label{cayley}

In 1845, 2 years after William Rowan Hamilton discovered quaternions, Arthur Cayley attempted to define the determinant of a quaternionic matrix using the usual formula (we denote the Cayley determinant by $Cdet$). Note that the following definition can be extended for $n\times n$ quaternionic matrices.

\begin{definition}[$2\times 2$ Cayley Determinant]
 For a $2 \times 2$ quaternionic matrix \newline $A = $ \begin{pmatrix} a & b \\ c & d \end{pmatrix}, $Cdet(A) = ad - cb$ for $a,b,c,d \in \HH$ \cite{aslaksen}. 
\end{definition}
 
 Notice that the order in which the entries are multiplied matters. We can see that for a $3\times 3$ quaternionic matrix 

 $B = $
\begin{pmatrix}
a & b & c \\
d & e & f \\
g & h & i
\end{pmatrix}, $Cdet(B) = (aei+bfg+cdh)-(gec+hfa+idb)$.
 
 The following examples are given in \cite{aslaksen}.
 \begin{ex} \label{singbutnot}
 	Let $M = $
 	\begin{pmatrix}
 		\kb & \jb \\
 		\ib & 1
 	\end{pmatrix}.
 	Then, $Cdet(M) = \kb-\ib\jb = \kb-\kb = 0$. Hence, we can say that by the Cayley determinant, $M$ is singular.
 \end{ex}
 \begin{ex}
 	Consider the transpose of the matrix $M$, $M^T = $
 	\begin{pmatrix} \label{invertbutnot}
 		\kb & \ib \\
 		\jb & 1
 	\end{pmatrix}.
 	Then $Cdet(M^T) = \kb-(\jb\ib) = \kb - (-\kb) = 2\kb$. Hence, we can say that by the Cayley determinant, $M^T$ is invertible.
 \end{ex}
  Taking into account the fact that the quaternions are non-commutative, one might ask whether or not this determinant behaves the way we expect - Will it really determine whether or not a quaternionic matrix is singular or not? Will the properties of the determinant still hold? Will the determinant still be a map from $M_{n}(G) \rightarrow G$ (in this case, $G = \HH$)? The last question comes from the fact that the determinants of complex matrices is a map from $\Mc{n} \rightarrow \CC$.  
}

\iffalse
\subsection{The Determinant Function}

We take a step back and revisit what it means for a mapping to be 	a determinant. We present the determinant function as defined by J.L. Brenner.

\begin{definition}[Determinant Function]
	For a field $F$, a determinant over the matrices of $\Mf{n}$ is a function $det$ from $\Mf{n}$ into $F$ such that 
	\begin{equation}
	det(AB) = det(A)det(B) = det(B)det(A)
	\end{equation} 
	holds either \textbf{(1)} $\forall A, B \in \Mf{n}$ or \textbf{(2)} $\forall$ invertible $A, B \in \Mf{n}$. 
\end{definition}

Notice that the definition requires the images of $A$ and $B$ to commute (this is not possible for skew-fields like the quaternions). We can see this matter viewed in a more rigorous manner (also in a manner more specific to the quaternions) while discussing Aslaksen's axioms. 

We see that if $det$ is a constant function that only maps to either 0 or 1 (with 0 for singular matrices and 1 for invertible matrices), then $det$ satisfies the above definition \cite{brenner}. The following theorem by Brenner shows that this holds for non-trivial determinants as well and that conditions (1) and (2) are essentially equivalent \cite{brenner}. 

\begin{theorem}
If $det$ is not constantly equal to 1 or 0 (i.e., $det$ is not a mapping $det: \Mf{n} \rightarrow F$ where $F$ is a field with two elements), then $det(B) = 0$ for all singular matrices. 
\end{theorem}

\begin{proof}
	Let $O$ be the zero matrix, $I$ the identity matrix, and $A$ a matrix whose determinant (image under $det$) is neither 1 nor 0. 
	Since $OA = O$, by Axiom 2, $det(O)det(A) = det(O)$. Suppose $det(O) \neq 0$, then $det(A) = 1$ which is a contradiction. Hence, $det(O) = 0$.
	Similarly, since $IA = A$, by Axiom 2, $det(I)det(A) = det(A) \implies det(I) = 1$.
	Also note that for a permutation matrix (an elementary matrix that permutes the rows/columns) $E$, $E^m = I$ for some $m$, hence $det(E) \neq 0$.
	Now, define a matrix
	\begin{equation*}
		D = 
		\begin{pmatrix}
			1 &  &  &  &  &  \\
			 & \ddots & & & & \\
			  & & 1 & & & \\
			  & & & 0 & & \\
			  & & & & \ddots & \\
			  & & & & & 0
		\end{pmatrix}
	\end{equation*}
	Notice that if $D$ is singular, then $det(D) = 0$.
\end{proof}
\fi

\subsection{Brenner's Determinant Function and Aslaksen's Axioms}

We take a step back and revisit what it means for a mapping to be 	a determinant. J.L. Brenner and Helmer Aslaksen offer different approaches to this, however, we will see how we can both arrive at the same conclusions. Brenner presented a general definition for a determinant function.
\begin{definition}[Determinant Function] \label{detfn}
	For a field $F$, a determinant over the matrices of $\Mf{n}$ is a function $det$ from $\Mf{n}$ into $F$ such that 
	\begin{equation}
	det(AB) = det(A)det(B) = det(B)det(A)
	\end{equation} 
	holds either \textbf{(1)} $\forall A, B \in \Mf{n}$ or \textbf{(2)} $\forall$ invertible $A, B \in \Mf{n}$. 
\end{definition} 

Aslaksen, on the other hand, uses a more axiomatic approach in defining a determinant, presenting three determinant \emph{axioms} which a determinant definition must satisfy in order for it to behave the way we expect, i.e., it has the properties we associate with determinants. 
\begin{itemize}
	\item \textbf{Axiom 1.} $det(A) = 0$ if and only if $A$ is singular.
	\item \textbf{Axiom 2.} $det(AB) = det(A)det(B)$ for all quaternionic matrices $A$ and $B$.
	\item \textbf{Axiom 3.} If $A'$ is obtained by adding a left-multiple of a row to another row or a right-multiple of a column to another column, then $det(A')=det(A)$ (as we have already encountered in linear algebra, this operation can be described by an elementary matrix \cite{aslaksen}).
\end{itemize}

Notice that a determinant is essentially a function that:
\begin{enumerate}
 \item Maps to 0 if a matrix is singular
 \item Preserves multiplication and,
 \item Remains unchanged after applying the elementary operation of adding a left/right-multiple of one row/column to another row/column respectively.
\end{enumerate}

Also notice that Alaksen's second axiom is the first condition in Brenner's determinant function.

We can trivially define the determinant merely as a function that constantly maps to 0 for all singular matrices and 1 for all non-singular matrices \cite{brenner}. This will easily satisfy the above axioms \cite{brenner} \cite{aslaksen}, however, we will mostly deal with determinants that are non-trivial (for instance the $\rdet$, $\ccdet$, and $Cdet$). 

\begin{theorem} \label{nontrivdetfn}
If $det$ is not constantly equal to 1 or 0 (i.e., $det$ is not a mapping $det: \Mf{n} \rightarrow F$ where $F$ is a field with two elements), then $det(B) = 0$ for all singular matrices. 
\end{theorem}

Theorem \ref{nontrivdetfn} not only shows how the determinant function in \ref{detfn} holds for the non-trivial case, it also shows that conditions (1) and (2) of the determinant function are essentially equivalent \cite{brenner}. 

\begin{theorem} \label{comh}
	If $det$ satisfies all of Aslaksen's axioms, then $det(\Mh{n})$ is a commutative subset of $\HH$. \cite{aslaksen}
\end{theorem}

Notice how theorem \ref{comh} is already implied in Brenner's determinant function in which it is already deemed necessary for the images to commute. By Theorem \ref{comh}, we see that $det$ cannot be a mapping onto $\HH$. Since $Cdet$ is onto $\HH$, by contrapositive of theorem \ref{comh}, $Cdet$ does not satisfy one of the axioms - in fact, it doesn't satisfy any of them \cite{aslaksen}. 

As an illustration (as seen in \cite{aslaksen}), recall examples \ref{singbutnot} and \ref{invertbutnot}. Notice that,
\begin{align*}
	&\text{\indent\indent\indent}M\begin{pmatrix}
	x \\
	y
	\end{pmatrix} = 
	\begin{pmatrix}
	0 \\
	0
	\end{pmatrix} \\ &\implies
	\begin{pmatrix}
	 		\kb & \jb \\
	 		\ib & 1
	\end{pmatrix}
	\begin{pmatrix}
	x \\
	y
	\end{pmatrix} = 
	\begin{pmatrix}
	0 \\
	0
	\end{pmatrix} \\ &\implies
	\kb x + \jb y = 0 \text{ and } \ib x + y = 0 \\ &\implies
	x = 0 \text{ and } y = 0. \\ &\implies
	M \text{ is invertible. This contradicts with the fact that $Cdet(M) = 0$.}
\end{align*}

Also notice that,

\begin{align*}
	&M^T\begin{pmatrix}
	-1 \\
	\jb
	\end{pmatrix} = 
	\begin{pmatrix}
	 		\kb & \ib \\
	 		\jb & 1
	\end{pmatrix}
	\begin{pmatrix}
	-1 \\
	\jb
	\end{pmatrix} = 
	\begin{pmatrix}
	0 \\
	0
	\end{pmatrix} \\ &\implies
	M^T \text{ is singular. This contradicts with the fact that $Cdet(M^T) = 2\kb$.}
\end{align*}

 Aslaksen provides counterexamples that show how the Cayley determinant fails to satisfy the other two axioms \cite{aslaksen}.

\section{Matrix Homomorphisms}

 It is clear that the classical definition of the determinant cannot be extended to quaternionic matrices. It was not until 1920, that a new approach in defining a quaternionic determinant was presented in a paper by Eduard Study \cite{aslaksen}. His idea involved transforming quaternionic matrices into complex matrices from which one could then just simply take the determinant \cite{aslaksen}. The method involves homomorphisms between quaternionic, complex, and real matrices. 

 In this section, we take a closer look into these homomorphisms - first discussing the motivation behind them and then the theory. 

\subsection{Representing Complex Numbers as Real Matrices}

In abstract algebra, we saw that we can define a bijection from the field of complex numbers to the 2D-plane ($\R^2$) - a mapping $\Theta : \CC \rightarrow \R^2$ where a complex number $a+b\ib$ is mapped to a vector/point $(a,b)$ in the 2D-plane. Therefore, in order to represent complex numbers as real matrices, we have to find a way to view them as linear maps in $\R^2$. 

Consider the complex function $f(z) = (a+b\ib)z$ (since $\CC$ is a field, $f$ is linear map). We see that the images of $1$ and $\ib$ are $a+b\ib$ and $-b+a\ib$ respectively. Under the function $\Theta$ (in which case $1$ is mapped to $(1,0)$ while $\ib$ is mapped to $(0,1)$), we seek a matrix in $\Mr{2}$ that maps $(1,0)$ to $\Theta(a+b\ib) = (a,b)$ and $(0,1)$ to $\Theta(-b+a\ib) = (-b,a)$. 
\\
\noindent Let this matrix be $F =$ 
\begin{pmatrix} 
\alpha & \beta \\ 
\chi & \delta 
\end{pmatrix} where $\alpha, \beta, \chi, \text{ and } \delta \in \R$. Then, 
\begin{align*}
	\begin{pmatrix} 
		\alpha & \beta \\ 
		\chi & \delta 
	\end{pmatrix} 
	\begin{pmatrix} 
		1 \\ 0 
	\end{pmatrix} &= 
	\begin{pmatrix} a \\ b \end{pmatrix} \implies
	\begin{pmatrix}
		\alpha \\ \chi
	\end{pmatrix} =
	\begin{pmatrix} a \\ b \end{pmatrix} \implies 
	\alpha = a; \chi = b \text{ and} \\
	\begin{pmatrix} 
		\alpha & \beta \\ 
		\chi & \delta 
	\end{pmatrix} 
	\begin{pmatrix} 
		0 \\ 1 
	\end{pmatrix} &= 
	\begin{pmatrix} -b \\ a \end{pmatrix} \implies
	\begin{pmatrix}
		\beta \\ \delta
	\end{pmatrix} =
	\begin{pmatrix} -b \\ a \end{pmatrix} \implies
	\beta = -b; \delta = a
\end{align*}
\noindent Therefore, $F$ = \begin{pmatrix} \alpha & \beta \\ \chi & \delta \end{pmatrix} = \begin{pmatrix} \label{phismall} a & -b \\ b & a \end{pmatrix}. The matrix $F$ can be seen as the matrix representation of the linear map $f$ which is defined by multiplying a complex number $z$ by $a+b\ib$. 

\begin{ex}
	Take the complex number $z = -3+2\ib$. Then its real matrix representation is
	\begin{pmatrix}
		-3 & -2 \\
		2 & -3
	\end{pmatrix}. 
	We can use this real matrix representation to multiply $z$ with another complex number, say $1+2\ib$. 
	\begin{pmatrix}
		-3 & -2 \\
		2 & -3
	\end{pmatrix}
	\begin{pmatrix}
		1 \\ 2
	\end{pmatrix}
	$ = $
	\begin{pmatrix}
		-7 \\
		-4
	\end{pmatrix}
	which corresponds to the complex number obtained by multiplying $z(1+2\ib) = (1+2\ib)z = -7-4\ib$.
\end{ex}

It is important to note that $F$ in general, doesn't represent the complex number $a+b\ib$ itself but the function associated with multiplying a complex number by $a+b\ib$. In this case, $F$ can represent both left and right multiplication because complex numbers are commutative under multiplication.

\subsection{Homomorphisms from $\Mc{n}$ to $\Mr{2n}$}

In order to represent complex matrices as real matrices, notice that every complex matrix can be represented as the sum of a real matrix and a purely imaginary matrix, i.e., for an $n\times n$ complex matrix $Z$, $Z = A + B\ib$ where $A,B \in \Mr{n}$ \cite{aslaksen}. We define a mapping 
\begin{equation*} 
\phi(A+B\ib) = 
\begin{pmatrix} 
A & -B \\ B & A 
\end{pmatrix}  
\cite{aslaksen}
\end{equation*} 

Notice how this mapping is simply a generalization of the matrix $F$ in \ref{phismall}. If we can show that $\phi$ is an injective homomorphism (i.e., it preserves multiplication and is one-to-one), we can essentially represent any complex matrix as a $2n \times 2n$ real matrix. Also notice how the dimension of the real matrix is necessarily even. This is a direct consequence of theorem \ref{dnc}.

Theorems \ref{distributive} and \ref{bigmatm} will come in handy in proving that $\phi$ is an injective homomorphism.

\begin{theorem}\label{distributive}
	For matrices $A,B,C \in \Mh{n}$, $A(B+C) = AB + AC$.
}
\end{theorem}

\begin{proof}
	Let $A = [a_{ij}]$, $B = [b_{ij}]$, $C = [c_{ij}] \in \Mc{n}$. Then $B+C = [b_{ij}+c_{ij}]$ and \begin{equation} 
	\begin{align*} 
	A(B+C) &= [\sum_{k=1}^{n}a_{ik}(b_{kj}+c_{kj})] = [\sum_{k=1}^{n}(a_{ik}b_{kj}+a_{ik}c_{kj})] \\ 
	&= [\sum_{k=1}^{n}a_{ik}b_{kj} + \sum_{k=1}^{n}a_{ik}c_{kj}] = [\sum_{k=1}^{n}a_{ik}b_{kj}] + [\sum_{k=1}^{n}a_{ik}c_{kj}] = AB + AC 
	\end{align*} \end{equation}
\end{proof}

The same method of proof can be used for the right distributive law. Furthermore, since $\R \subseteq \CC \subseteq \HH$, theorem \ref{distributive} holds for matrices in $\Mr{n}$ and $\Mc{n}$ (we can show this by seeing that the matrices in $\Mh{n}$ form a ring). 

\begin{theorem} \label{bigmatm}
	For matrices $A_{ij}, B_{ij} \in \Mh{n}$ where $i,j = 1,2,...,m$, 
	\begin{equation*}
		\pgenmat{A}\pgenmat{B} = 
		\begin{pmatrix}
			\sum_{k=1}^{m}A_{1k}B_{k1} & \cdots & \sum_{k=1}^{m}A_{1k}B_{km} \\
			\vdots & \ddots & \vdots \\
			\sum_{k=1}^{m}A_{mk}B_{k1} & \cdots & \sum_{k=1}^{m}A_{mk}B_{km} 
		\end{pmatrix}
	\end{equation*}
\end{theorem}
\newcommand{\gengmat}[1]{\pgenmatk{\genmat{#1_{11}}}{\genmat{#1_{1m}}}{\genmat{#1_{m1}}}{\genmat{#1_{mm}}}}
\newcommand{\ddsumprod}[2]{\sum_{k=1}^m[\sum_{l=1}^na_{#1 il}b_{#2 lj}]}
\begin{proof}
	\begin{align*}
		&\pgenmat{A}\pgenmat{B} \\
		=&\gengmat{a}\gengmat{b} \\
	%\end{align*}
	%\begin{align*}
		=&\pgenmatk{\ddsumprod{1k}{k1}}{\ddsumprod{1k}{km}}{\ddsumprod{mk}{k1}}{\ddsumprod{mk}{km}} \\
		=&\begin{pmatrix}
			\sum_{k=1}^{m}A_{1k}B_{k1} & \cdots & \sum_{k=1}^{m}A_{1k}B_{km} \\
			\vdots & \ddots & \vdots \\
			\sum_{k=1}^{m}A_{mk}B_{k1} & \cdots & \sum_{k=1}^{m}A_{mk}B_{km} 
		\end{pmatrix}
	\end{align*}
\end{proof}

Theorem \ref{bigmatm} will make it convenient for us to multiply square matrices with square matrices as entries. Again, this theorem also holds for matrices in $\Mr{n}$ and $\Mc{n}$ because $\R \subseteq \CC \subseteq \HH$. 

\begin{theorem} \label{phimorph}
	Let $\phi: \Mc{n} \rightarrow \Mr{2n}$ such that $C+D\ib \mapsto $ \begin{pmatrix} C & -D \\ D & C \end{pmatrix} where $C+D\ib \in \Mc{n}$. Then $\phi$ is an injective homomorphism. 
\end{theorem}
\newpage
\begin{proof}
	\textbf{\newline1-1:}
	\begin{align*}
		&\phi(A+B\ib) = \phi(C+D\ib) \\
		&\implies 
		\begin{pmatrix}
		A & -B \\ 
		B & A 
		\end{pmatrix} = 
		\begin{pmatrix}
		C & -D \\ 
		D & C 
		\end{pmatrix} \\
		&\implies A = C \text{ and } B = D \text{ by Matrix Equality} \\
		&\implies A+B\ib = C+D\ib \\
		&\implies \phi \text{ is injective.}
	\end{align*}
	\textbf{Homomorphism: \newline}
	Let $A+B\ib$, $C+D\ib \in \Mc{n}$. Then 
	\begin{align*}
		\phi[(A+B\ib)(C+D\ib)] &= \phi[(A+B\ib)C+(A+B\ib)D\ib] \text{ by Theorem \ref{distributive}} \\
		&= \phi[AC+BC\ib+AD\ib-BD] \\
		&= \phi[(AC-BD)+(BC+AD)\ib] \\
		&= 
		\begin{pmatrix} 
		AC-BD & -(BC+AD) \\ 
		BC+AD & AC-BD 
		\end{pmatrix}
	\end{align*}
	\begin{align*}
		\phi[(A+B\ib)]\phi[(C+D\ib)] &= 
		\begin{pmatrix}
		A & -B \\ 
		B & A 
		\end{pmatrix}
		\begin{pmatrix}
		C & -D \\ 
		D & C 
		\end{pmatrix} \\
		&=
		\begin{pmatrix} 
		AC-BD & -(BC+AD) \\ 
		BC+AD & ac-bd 
		\end{pmatrix} \text{ by Theorem \ref{bigmatm}.}
	\end{align*}
\end{proof}
\newline

Alternatively, we can define a matrix 
\begin{equation*} 
J = 
\begin{pmatrix} 
0 & -I_n \\ 
I_n & 0 
\end{pmatrix} 
\end{equation*}
 Notice that the matrix $J$ is the image of $iI \in \Mc{n}$ under $\phi$ \cite{aslaksen}. Intuitively, this means that $J$ represents the imaginary number $\ib$ in $\Mr{2n}$. It can be easily shown that $J^2 = -I$. Therefore, $J$ gives a \emph{complex structure} in $\R^{2n}$. Recall in Chapter 2 that in order for a subset of matrices in $\Mr{2n}$ to "mimic" a complex vector space, every linear map in the said subset must commute with the complex structure. We see that the subset of images of complex matrices under $\phi$ satisfy this. We can then write $\phi(\Mc{n}) = \{P \in \Mr{2n} | JP = PJ\}$\cite{aslaksen}. 

 \begin{ex}
 	Take the complex matrix 

 	$Z = $
 	\begin{pmatrix}
 		1+2\ib & 3\ib \\
 		1 & -2+\ib
 	\end{pmatrix} $ = $
 	\begin{pmatrix}
 		1 & 0 \\
 		1 & -2
 	\end{pmatrix} $+$
 	\begin{pmatrix}
 		2 & 3 \\
 		0 & 1
 	\end{pmatrix} $\ib$.
 	Then, $\phi(Z) = $
 	\begin{pmatrix}
 		\begin{matrix}
 		1 & 0 \\
 		1 & -2
 		\end{matrix} & 
 		\begin{matrix}
 		-2 & -3 \\
 		1 & -1
 		\end{matrix} \\
 		\begin{matrix}
 		2 & 3 \\
 		0 & 1
 		\end{matrix} &
 		\begin{matrix}
 		1 & 0 \\
 		1 & -2
 		\end{matrix}
 	\end{pmatrix}
 \end{ex}

\subsection{Representing Quaternions as Complex Matrices} \label{qrep}

Recall in Chapter 2 that we can view the quaternions as a 2-dimensional algebra over $\CC$ by having $\quat{a}{b}{c}{d} = (a+b\ib)+\jb(c-d\ib)$. In general, we can write any quaternion as $x+\jb y$ where $x, y \in \CC$. Because of this, we see that we can define a bijection $\Omega: \HH \rightarrow \CC^2$ where a quaternion $q = x+\jb y$ is mapped to $(x, y) \in \CC^2$. In order to represent quaternions as complex matrices, we have to find a way to view them as linear maps in $\CC^2$.

Let us consider the quaternionic function $s(q) = (x+\jb y)q$. We see that,
\begin{align*}
	&s(1) = x+\jb y \\
	&s(\ib) = (x+\jb y)\ib = x\ib + \jb (y\ib) \\
	&s(\jb) = (x+\jb y)\jb = x\jb + \jb y\jb = \jb\bar{x}+\jb^2 \bar{y} = -\bar{y}+\jb\bar{x} \\
	&s(\kb) = (x+\jb y)\kb = x\kb + \jb y\kb = -x\jb\ib + \bar{y}\jb\kb = \bar{y}\ib - \jb(\bar{x}\ib) 
\end{align*}

\newcommand{\kmat}{\begin{pmatrix} \kappa & \lambda \\ \mu & \nu \end{pmatrix}}
\newcommand{\krmat}{\begin{pmatrix} \kappa_R & \lambda_R \\ \mu_R & \nu_R \end{pmatrix}}
\newcommand{\vectC}[2]{\begin{pmatrix} #1 \\ #2 \end{pmatrix}}

Under the function $\Omega$, we see that the images of 1, $\ib, \jb$ and $\kb$ are $(1,0) , (\ib,0), (0,1)$ and $(0,-\ib)$ respectively and the images of $s(1), s(\ib), s(\jb)$ and $s(\kb)$ are $(x,y), (x\ib,y\ib), (-\bar{y},\bar{x})$ and $(\bar{y}\ib,-\bar{x}\ib)$. We seek a matrix in $\Mc{2}$ such that $(1,0) \mapsto (x,y), (\ib,0) \mapsto (x\ib,y\ib), (0,1) \mapsto (-\bar{y},\bar{x})$, and $(0,-\ib) \mapsto (\bar{y}\ib,-\bar{x}\ib)$.

 Let this matrix be $S = \kmat$ , where $\kappa, \lambda, \mu, $ and $\nu \in \CC$. Then, 

\begin{align*}
	&\kmat \vectC{1}{0} = \vectC{x}{y} \implies \vectC{\kappa}{\mu} = \vectC{x}{y} \implies \kappa = x \text{ and } \mu = y \\
	&\kmat \vectC{\ib}{0} = \vectC{x\ib}{y\ib} \implies \vectC{\kappa\ib}{\mu\ib} = \vectC{x\ib}{y\ib} \implies \kappa = x \text{ and } \mu = y \\
	&\kmat \vectC{0}{1} = \vectC{-\bar{y}}{\bar{x}} \implies \vectC{\lambda}{\nu} = \vectC{-\bar{y}}{\bar{x}} \implies \lambda = -\bar{y} \text{ and } \nu = \bar{x} \\
	&\kmat \vectC{0}{-\ib} = \vectC{\bar{y}\ib}{-\bar{x}\ib} \implies \vectC{-\lambda\ib}{-\nu\ib} = \vectC{\bar{y}\ib}{-\bar{x}\ib} \implies \lambda = -\bar{y} \text{ and } \nu = \bar{x}
\end{align*}

Hence, $S = \kmat = $
\begin{pmatrix}
	x & -\bar{y} \\
	y & \bar{x}
\end{pmatrix}.
The matrix $S$ can be seen as the complex matrix representation of the quaternionic linear map $s$ which is defined by performing a left-multiplication by a quaternion $x+\jb y$. Notice that we only need to look into the images of 1 and $\jb$ (or $\ib$ and $\kb$) to obtain $S$. This is because their images under the function $\Omega$ form a basis for $\CC^2$. Also notice that $s$ is a linear transform. This is why we can obtain a matrix representation of $s$ in $\CC$.

\begin{ex}
	Take the quaternion $q = -2+\ib-5\jb+2\kb = -2+\ib + \jb(-5-2\ib)$. Then its complex matrix representation is 
	\begin{pmatrix}
		-2+\ib & 5-2\ib \\
		-5-2\ib & -2-\ib
	\end{pmatrix}.
	We can use this matrix to multiply $q$ with another quaternion, say $1+\ib+3\jb-4\kb = 1+\ib + \jb(3+4\ib)$. We have, 
	\begin{pmatrix}
		-2+\ib & 5-2\ib \\
		-5-2\ib & -2-\ib
	\end{pmatrix}
	$\vectC{1+\ib}{3+4\ib} = \vectC{20+13\ib}{-5-18\ib}$ which corresponds to the quaternion obtained by multiplying $q(1+\ib+3\jb-4\kb) = 20+13\ib-5\jb+18\kb = 20+13\ib+\jb(-5-18\ib)$ on the left.
\end{ex}
Notice that $S$ only represents left multiplication in the quaternions. One would probably ask whether or not we can have a complex matrix representation for right multiplication.
\newline
Consider the quaternionic function defined by multiplying a quaternion $q$ by some quaternion $x+\jb y$ on the right, $s_R(q) = q(x+\jb y)$ such that $y\neq 0$. We see that,
\begin{align*}
	s_R(1) &= x+\jb y \\
	s_R(\ib) &= \ib x - \jb\ib y \\
	s_R(\jb) &= -y + \jb x \\
	s_R(\kb) &= -\ib y -\jb\ib x 
\end{align*}

We use the same method in obtaining the complex matrix representation. We let $S_R = $ \krmat be the complex matrix representation of this function. 

\begin{align*}
	&\krmat \vectC{1}{0} = \vectC{x}{y} \implies \vectC{\kappa_R}{\mu_R} = \vectC{x}{y} \implies \kappa_R = x \text{ and } \mu_R = y \\
	&\krmat \vectC{\ib}{0} = \vectC{\ib x}{-\ib y} \implies \vectC{\kappa_R\ib}{\mu_R\ib} = \vectC{\ib x}{-\ib y} \implies \kappa_R = x \text{ and } \mu_R = -y \\ 
	&\krmat \vectC{0}{1} = \vectC{-y}{x} \implies \vectC{\lambda_R}{\nu_R} = \vectC{-y}{x} \implies \lambda_R = -y \text{ and } \nu_R = x \\
	&\krmat \vectC{0}{-\ib} = \vectC{-\ib y}{-\ib x} \implies \vectC{-\lambda_R\ib}{-\nu_R\ib} = \vectC{-\ib y}{-\ib x} \implies \lambda_R = y \text{ and } \nu_R = x 
\end{align*}

This implies that $\mu_R = \lambda_R = 0$. Hence, the complex matrix representation of right multiplication by a quaternion is 
\begin{pmatrix}
	x & 0 \\
	0 & x
\end{pmatrix}.
However, we see that, 
\begin{pmatrix}
	x & 0 \\
	0 & x
\end{pmatrix}
\vectC{1}{0} $ = $ \vectC{x}{0} which doesn't correspond to $s_R(1) &= x+\jb y$. We will later see that there is no matrix representation for right multiplication by a quaternion because it is not a linear transform in $\HH$. However, notice that every linear transform in $\HH$ commutes with right multiplication, i.e., 
\begin{pmatrix}
	x & -\bar{y} \\
	y & \bar{x}
\end{pmatrix} 
\begin{pmatrix}
	x & 0 \\
	0 & x
\end{pmatrix} $ = $
\begin{pmatrix}
	x & 0 \\
	0 & x
\end{pmatrix}
\begin{pmatrix}
	x & -\bar{y} \\
	y & \bar{x}
\end{pmatrix}. We will discuss this matter further in the next subsection.

\subsection{Homomorphisms from $\Mh{n}$ to $\Mc{2n}$}

Before proceeding, let's first generalize the matrix $S$ and define a mapping from $\Mh{n}$ to $\Mc{2n}$. In order to represent quaternionic matrices as complex matrices, notice that every quaternionic matrix can be represented as the sum $Q = X + \jb Y$ where $X, Y \in \Mc{n}$ \cite{aslaksen}. We define a mapping \begin{equation*} \psi(X+\jb Y) = 
\begin{pmatrix} 
X & -\overline{Y} \\ 
Y & \overline{X} 
\end{pmatrix}  
\cite{aslaksen}
\end{equation*} 

Again, if we can show that $\psi$ is an injective homomorphism, we can essentially represent any quaternionic matrix as a $2n \times 2n$ complex matrix.

\begin{theorem} \label{psimorph}
 	Let $\psi: \Mh{n} \rightarrow \Mc{2n}$ such that $X+\jb Y \mapsto $ 
 	\begin{pmatrix} 
 	X & -\overline{Y} \\ 
 	Y & \overline{X} 
 	\end{pmatrix} 
 	where $X+\jb Y \in \Mh{n}$. Then $\psi$ is an injective homomorphism. 
\end{theorem}

\begin{proof}
	\textbf{\newline1-1:}
	\begin{align*}
		&\psi(X+\jb Y) = \psi(V+\jb W) \\
		&\implies 
		\begin{pmatrix}
		X & -\overline{Y} \\ 
 		Y & \overline{X} 		
 		\end{pmatrix} = 
		\begin{pmatrix}
		V & -\overline{W} \\ 
 		W & \overline{V} 
		\end{pmatrix} \\
		&\implies X = V \text{ and } Y = W \text{ by Matrix Equality} \\
		&\implies X+\jb Y = V+\jb W \\
		&\implies \psi \text{ is injective.}
	\end{align*}
	\textbf{\newpage Homomorphism: \newline}
	Let $X+\jb Y$, $V+\jb W \in \Mh{n}$. Then 
	\begin{align*}
		\psi[(X+\jb Y)(V+\jb W)] &= \psi[X(V+\jb W)+\jb Y(V+\jb W)] \text{ by Theorem \ref{distributive}} \\
		&= \psi[XV+X\jb W + \jb YV+ \jb Y\jb W] \\
		&= \psi[XV+\jb \overline{X}W + \jb YV + \jb^2\overline{Y}W] \\
		&= \psi[(XV-\overline{Y}W)+\jb(\overline{X}W+YV)] \\
		&=
		\begin{pmatrix} 
		XV-\overline{Y}W & -\overline{(\overline{X}W+YV)} \\ 
		\overline{X}W+YV & \overline{XV-\overline{Y}W} 
		\end{pmatrix} \\
		&=
		\begin{pmatrix} 
		XV-\overline{Y}W & -X\overline{W}-\overline{Y}\overline{V} \\ 
		\overline{X}W+YV & \overline{X}\overline{V}-Y\overline{W} 
		\end{pmatrix}
		\end{align*}
		\begin{align*}
		\psi[(X+\jb Y)]\psi[(V+\jb W)] &= 
		\begin{pmatrix}
		X & -\overline{Y} \\ 
		Y & \overline{X} 
		\end{pmatrix}
		\begin{pmatrix}
		V & -\overline{W} \\ 
		W & \overline{V} 
		\end{pmatrix} \\
		&=
		\begin{pmatrix} 
		XV-\overline{Y}W & -X\overline{W}-\overline{Y}\overline{V} \\ 
		\overline{X}W+YV & \overline{X}\overline{V}-Y\overline{W} 
		\end{pmatrix} \text{ by Theorem \ref{bigmatm}.}
	\end{align*}
\end{proof}

The non-commutativity of quaternions presents some problems in representing \emph{quaternionic linear maps} as complex linear maps. If we consider a quaternionic linear map say $L(v) = Av$ for $A \in \Mh{n}$ where we take in quaternions as scalars, then, $cAv = c L(v) = L(cv) = Acv$ which is false \cite{stack}. However, $Avc = L(v)c = L(vc) = Avc$. Hence, we now see that any quaternionic linear map commutes with right scalar multiplication by a quaternion (as was seen in subsection \ref{qrep}). However, right scalar multiplication itself is not a linear map in $\HH$ (in order for it to be a linear map, it in turn, has to commute with other quaternions)\cite{stack} \cite{aslaksen}. This poses a problem because it implies that there is no matrix representation for right scalar multiplication \cite{aslaksen}. 

However, in \cite{aslaksen}, we see that we can consider a linear map $\widetilde{R_j}$ in $\CC^{2n}$ as the image of right scalar multiplication by $\jb$ under the homomorphism. $\widetilde{R_j}$ corresponds to multiplying $v \in \CC^{2n}$ by the matrix $J$ and then conjugating \cite{aslaksen}. This gives a quaternionic structure in $\CC^{2n}$ and thus, a quaternionic linear map corresponds to a complex linear map $Q$ that commutes with $\widetilde{R_j}$, i.e., $Q \overline{Jv} = \overline{JQv}$ for $v \in \CC^{2n}$. It can be easily shown that the latter holds if and only if $\overline{Q}J = JQ$ using the fact that $Q\overline{Jv} = \overline{QJv}$. Thus, $\psi(\Mh{n}) = \{Q \in \Mc{2n} |\overline{Q}J = JQ\}.

\begin{ex}
	Take the quaternionic matrix 

	$Q = $
	\begin{pmatrix}
		1+2\ib-3\jb+\kb & 2\ib+5\kb \\
		1-\ib & 3+\jb+\kb
	\end{pmatrix} $ = $
	\begin{pmatrix}
		1+2\ib & 2\ib \\
		1-\ib & 3
	\end{pmatrix} $+\jb$
	\begin{pmatrix}
		-3-\ib & -5\ib \\
		0 & 1-\ib
	\end{pmatrix}.

	Hence, $\psi(Q) = $
	\begin{pmatrix}
		\begin{matrix}
			1+2\ib & 2\ib \\
			1-\ib & 3
		\end{matrix} &
		\begin{matrix}
			3-\ib & -5\ib \\
			0 & -1-\ib
		\end{matrix} \\
		\begin{matrix}
			-3-\ib & -5\ib \\
			0 & 1-\ib
		\end{matrix} &
		\begin{matrix}
			1-2\ib & -2\ib \\
			1+\ib & 3
		\end{matrix} 
	\end{pmatrix}.
\end{ex}

\section{The Study Determinant}

\begin{definition}[Study Determinant]
	For $M \in \Mh{n}$, the \emph{Study Determinant} is defined by $Sdet M = \cdet{\psi(M)} = \sqrt{\rdet{\phi(\psi(M))}}.
\end{definition}

Notice that we can compute for the Study determinant in two different ways: 
\begin{enumerate}
	\item Simply getting the complex matrix representation of the quaternionic matrix and proceeding to take its complex determinant or;
	\item Getting the real matrix representation of the quaternionic matrix by composing $\phi$ and $\psi$, and then taking the square root of the real determinant.
\end{enumerate}

In the following example, we take the Study determinant of the quaternionic matrix in Example \ref{singbutnot} for which the Cayley determinant failed to identify as non-singular. 

\begin{ex}
	Let $M = $ 
	\begin{pmatrix}
 		\kb & \jb \\
 		\ib & 1
 	\end{pmatrix} $ = $
 	\begin{pmatrix}
 		0 & 0 \\
 		\ib & 1
 	\end{pmatrix} $+\jb$
 	\begin{pmatrix}
 		-\ib & 1 \\
 		0 & 0
 	\end{pmatrix}.
 	Then,

 	\begin{enumerate}
 		\item $Sdet(M) = \ccdet(\psi(M)) = \ccdet$
		\begin{pmatrix}
			\begin{matrix}
				0 & 0 \\
				\ib & 1
			\end{matrix} &
			\begin{matrix}
				-\ib & -1 \\
				0 & 0
			\end{matrix} \\
			\begin{matrix}
 				-\ib & 1 \\
 				0 & 0
 			\end{matrix} &
 			\begin{matrix}
				0 & 0 \\
				-\ib & 1
			\end{matrix}
 		\end{pmatrix} $ = 4$. 
 		\item We see that $\psi(M) = $
 		\begin{pmatrix}
 			0 & 0 & 0 & -1 \\
 			0 & 1 & 0 & 0 \\
 			0 & 1 & 0 & 0 \\
 			0 & 0 & 0 & 1
 		\end{pmatrix} $+$
 		\begin{pmatrix}
 			0 & 0 & -1 & 0 \\
 			1 & 0 & 0 & 0 \\
 			-1 & 0 & 0 & 0 \\
 			0 & 0 & -1 & 0
 		\end{pmatrix} $\ib$.

 		$\sqrt{\rrdet(\phi(\psi(M)))} =$
 		$\begin{pmatrix}
 		\rrdet
 		\begin{pmatrix}
 			0 & 0 & 0  & -1 &  0 & 0 & 1 &  0 \\
 			0 & 1 & 0  &  0 & -1 & 0 & 0 &  0 \\
 			0 & 1 & 0  &  0 &  1 & 0 & 0 &  0 \\
 			0 & 0 & 0  &  1 &  0 & 0 & 1 &  0 \\
 			0 & 0 & -1 &  0 &  0 & 0 & 0 & -1 \\
 			1 & 0 & 0  &  0 &  0 & 1 & 0 &  0 \\
 		   -1 & 0 & 0  &  0 &  0 & 1 & 0 &  0 \\
 		    0 & 0 & -1 &  0 &  0 & 0 & 0 &  1 \\
 		\end{pmatrix}
 		\end{pmatrix}^{1/2} = \sqrt{16} = 4$
 	\end{enumerate}


\end{ex}

Hence, we can see that by the Study determinant, the matrix $M$ is invertible which it really is as shown in section \ref{cayley}. 

Aslaksen has shown that the Study Determinant satisfies all of his axioms \cite{aslaksen}. Furthermore, he also showed that $Sdet(M) \geq 0$ \cite{aslaksen}.

\section{Skew-Coninvolutory Quaternionic Matrices of Odd Dimensions}
\begin{prop}[Main Result]
 $\mathscr{D}_n(\HH)$ is empty when $n$ is odd.
\end{prop}
\begin{proof}
	If $F \in \mathscr{D}_{n}(\HH)$ then $F\bar{F} = -I_n$. \newline Taking the Study determinant of both sides, 
	\begin{align*}
		Sdet(E\bar{E}) &= Sdet(-I_n) \\
		Sdet(E)Sdet(\bar{E}) &= (-1)^n \\
		Sdet(E)\overline{Sdet(E)} &= (-1)^n \text{, by Theorem \ref{detbar}} \\
		|Sdet(E)|^2 &= (-1)^n
	\end{align*}
	Since $|Sdet(E)|^2 > 0$, $(-1)^n > 0$. Hence, $n$ must be even.
\end{proof}

	Theorem \ref{detbar} applies because the Study determinant can be viewed as a complex determinant.

\begin{ex}
	Again consider the $1 \times 1$ case where, $\jb\bar{\jb} = -\jb^2 = 1 \neq -1$. Whereas in the $2 \times 2$ case, consider $E = $ 
	\begin{pmatrix}
		0 & \jb \\
		-\jb & 0
	\end{pmatrix}. Then, $E\overline{E} = $
	 \begin{pmatrix}
		0 & \jb \\
		-\jb & 0
	\end{pmatrix}
	\begin{pmatrix}
		0 & -\jb \\
		\jb & 0
	\end{pmatrix} $ = $
	\begin{pmatrix}
		-1 &  0 \\
		 0 & -1
	\end{pmatrix}.
\end{ex}