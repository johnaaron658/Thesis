\section{Complex Matrices}

\begin{definition}[Conjugate Matrix]
	A \emph{conjugate matrix} is a matrix $\bar{E}$ obtained from $E$ by taking the complex conjugate of every entry of $E$.
\end{definition}

\begin{definition}[Coninvolutory Matrix]
	A matrix is said to be \emph{coninvolutory} if $E\bar{E} = I_n$ for $E \in \Mc{n}$.
\end{definition}

\begin{remark}
	By manipulation, we obtain $E^{-1} = \bar{E}$. Hence, we may also say that a matrix whose inverse is its own conjugate matrix is a coninvolutory matrix. Furthermore, we see that coninvolutory matrices are the extension of complex numbers with modulus 1 \cite{stamaria}.
\end{remark}

\begin{definition}[Skew-Coninvolutory Matrix]
	A matrix is said to be \emph{skew-coninvolutory} if $E\bar{E} = -I_n$ for $E \in \Mc{n}$.
\end{definition}

\begin{remark}
	Again, we may say that a matrix whose inverse is the negative of its own conjugate matrix is a skew-coninvolutory matrix. This is analogous to the skew-symmetric matrices we've encountered in linear algebra.
\end{remark}

\begin{theorem} \label{detbar}
	For a matrix $E \in \Mc{n}$, $det(\bar{E}) = \overline{det(E)}$.
}
\end{theorem}

\begin{proof}
	We prove by mathematical induction. 
	\newline
	\newline \textbf{Base Case}: 
	For $E$ = 
	\begin{pmatrix}
		\bar{a} & \bar{b} \\
		\bar{c} & \bar{d}
	\end{pmatrix}, 
	$det(\bar{E})$ & =
	\begin{vmatrix}
		\bar{a} & \bar{b} \\
		\bar{c} & \bar{d}
	\end{vmatrix} $= \overline{ad} - \overline{bc} = \overline{ad-bc} = \overline{det(E)}$
	\newline
	\newline
	\textbf{Induction Hypothesis}:
	Suppose $det(\bar{E}) = \overline{det(E)}$ holds for $E \in \Mc{n}$.
	\newline
	Let $X \in \Mc{n+1}$. Then, $$det(\overline{X}) = \overline{\sum_{j=1}^{n+1} a_{ij}c_{ij}} = \sum_{j=1}^{n+1} \overline{a_{ij}}\text{ }\overline{c_{ij}}$$ is the $i^{th}$ row expansion of an $(n+1)\times (n+1)$ matrix where $\overline{c_{ij}}$ is the cofactor of $\overline{a_{ij}}$.
	\newline
	Note that $\overline{c_{ij}} = (-1)^{i+j}\overline{M_{ij}}$ where $\overline{M_{ij}}$ is the determinant of the $n\times n$ matrix obtained by deleting the $i^{th}$ row and the $j^{th}$ column of the original matrix.
	\newline
	By I.H., $\overline{M_{ij}}$ is the determinant of an $n\times n$ conjugate matrix. Thus, we see that we are computing for the determinant of an $(n+1)\times (n+1)$ conjugate matrix. 
\end{proof}

\subsection{Skew-Coninvolutory Complex Matrices}

We now show and prove a result concerning whether or not $\mathscr{D}_n(\C)$ is empty when $n$ is odd as seen in \cite{stamaria}.

\begin{theorem}
	$\mathscr{D}_{n}(\C)$ is empty when $n$ is odd.
\end{theorem}

\begin{proof}
	If $E \in \mathscr{D}_{n}(\C)$ then $E\bar{E} = -I_n$. \newline Taking the determinant of both sides, 
	\begin{align*}
		det(E\bar{E}) &= det(-I_n) \\
		det(E)det(\bar{E}) &= (-1)^n \\
		det(E)\overline{det(E)} &= (-1)^n \text{, by Theorem \ref{detbar}} \\
		|det(E)|^2 &= (-1)^n
	\end{align*}
	Since $|det(E)|^2 > 0$, $(-1)^n > 0$. Hence, $n$ must be even.
\end{proof}

\section{Quaternion Basics}

\subsection{Multiplication and Addition}

Recall in Chapter 1 - the four-dimensional algebra of quaternions is generated by the basis elements $\{1,\ib,\jb,\kb\}$ such that 
\begin{equation} \label{quat_eq}
\ib^2 = \jb^2 = \kb^2 = \ib \jb \kb = -1
\end{equation}

From the above equation, we can easily derive the following:
\begin{align*}
	\jb\kb &= \ib & \kb\jb &= -\ib \\
	\kb\ib &= \jb & \ib\kb &= -\jb \\
	\ib\jb &= \kb & \jb\kb &= -\kb
\end{align*}

Notice that the quaternions are not commutative under \emph{multiplication}. In general, for quaternions $q_1 = a_1 + b_1\ib + c_1\jb + d_1\kb$ and $q_2 = a_2 + b_2\ib + c_2\jb + d_2\kb$, 
\begin{align*}
	q_1q_2 &= (a_1a_2 - b_1b_2 - c_1c_2 - d_1d_2) + (a_1b_2 + b_1a_2 + c_1d_2 - d_1c_2)\ib \\
		   &+ (a_1c_2 - b_1d_2 + c_1a_2 + d_1b_2)\jb + (a_1d_2 + b_1c_2 - c_1b_2 + d_1a_2)\kb
\end{align*}

%insert vector representation

Quaternions are, however, commutative under \emph{addition} where $q_1+q_2 = (a_1+a_2) + (b_1+b_2)\ib + (c_1+c_2)\jb + (d_1+d_2)\kb$.

\subsection{Other Operations and Properties}

\begin{definition}[$\HH$-Conjugate]
	The $\HH$-Conjugate of a quaternion $q = \quat{a}{b}{c}{d}$ is $\bar{q} = a - b\ib - c\jb - d\kb$.
\end{definition}

\begin{remark}
	Notice that $q\bar{q} = (\quat{a}{b}{c}{d})(a - b\ib -c\jb-d\kb) = a^2+b^2+c^2+d^2$.
\end{remark}

\begin{definition}[$\HH$-Norm]
	The $\HH$-Norm of a quaternion $q = \quat{a}{b}{c}{d}$ is $|q| = \sqrt{q\bar{q}} = \sqrt{a^2+b^2+c^2+d^2}$
\end{definition}

\begin{definition}[Inverse]
	The inverse of a quaternion $q$ is $q^{-1}$ such that $q^{-1}q = qq^{-1} = 1$.
\end{definition}

\begin{theorem}
For $q, p, r \in \HH$,
	\begin{enumerate}
		\item $|q|^2 = q\bar{q}$.
		\item If $q\neq 0$, then $q^{-1} = \bar{q}/|q|^2$.
		\item $\overline{qp} = \bar{p}\bar{q}$.
		\item $(qp)^{-1} = p^{-1}q^{-1}$ provided that the inverses of $p$ and $q$ exist.
		\item $(qp)r = q(pr)$ that is, quaternion multiplication is associative.
	\end{enumerate}
\end{theorem}

%insert proof here.

\begin{remark}
	Notice that most of the properties we see in quaternions are merely extensions of the properties we see in complex numbers. 
\end{remark}

\subsection{Quaternionic Matrices}

Most of the definitions we've already mentioned for complex matrices can also be extended in the context of quaternionic matrices.

\begin{definition}[Conjugate Quaternionic Matrix]
	A \emph{conjugate quaternionic matrix} is a matrix $\bar{E}$ obtained from $E$ by taking the $\HH$-conjugate of every entry of $E$.
\end{definition}

\begin{definition}[Skew-Coninvolutory Quaternionic Matrix]
	A quaternionic matrix $E$ is said to be \emph{Skew-Coninvolutory} if $E\bar{E} = -I_n$.
\end{definition}

\section{Matrix Homomorphisms}

We look into functions that make it possible for us to represent complex numbers and quaternions as matrices. These functions are of extreme importance as they are used to define some of the quaternionic determinants we will encounter.

\subsection{Representing Complex Numbers as Real Matrices}

In abstract algebra, we saw that we can define an isomorphism from the field of complex numbers to the 2D-plane ($\R^2$) - a mapping $\phi : \C \rightarrow \R^2$ where a complex number $a+b\ib$ is mapped to a vector/point $(a,b)$ in the 2D-plane. Therefore, in order to represent complex numbers as real matrices, we have to find a way to view them as linear transformations over $\R^2$. 

Consider the complex function $f(z) = (a+b\ib)z$. We see that the images of $1$ and $\ib$ are $a+b\ib$ and $-b+a\ib$ respectively. 

\subsection{Homomorphisms from $\Mc{n}$ to $\Mr{2n}$}

\subsection{Representing Quaternions as Complex Matrices}

%exposition

\subsection{Homomorphisms from $\Mh{n}$ to $\Mc{2n}$}


\section{Quaternionic Determinants}

\subsection{Aslaksen's Axioms and the Cayley Determinant}

\subsection{The Study Determinant}

\subsection{The Dieudonne Determinant}

\subsection{Moore's Determinant}

%A quaternion $q = a + b\ib + c\jb + d\kb$ can be written as $q = (a+b\ib) + (c+d\ib)\jb = (a+b\ib) + (d-c\ib)\kb$ where $a+b\ib,c+d\ib,d-c\ib \in \C$. We can therefore view the set of quaternions as a two-dimensional algebra over \C \cite{stamaria}.
