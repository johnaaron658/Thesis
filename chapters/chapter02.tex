\section{Complex Matrices}

\begin{definition}[Conjugate Matrix]
	A \emph{conjugate matrix} is a matrix $\bar{E}$ obtained from $E$ by taking the complex conjugate of every entry of $E$.
\end{definition}

\begin{definition}[Coninvolutory Matrix]
	A matrix is said to be \emph{coninvolutory} if $E\bar{E} = I_n$ for $E \in \Mc{n}$.
\end{definition}

\begin{remark}
	By manipulation, we obtain $E^{-1} = \bar{E}$. Hence, we may also say that a matrix whose inverse is its own conjugate matrix is a coninvolutory matrix. 
\end{remark}

\begin{definition}[Skew-Coninvolutory Matrix]
	A matrix is said to be \emph{skew-coninvolutory} if $E\bar{E} = -I_n$ for $E \in \Mc{n}$.
\end{definition}

\begin{remark}
	Again, we may say that a matrix whose inverse is the negative of its own conjugate matrix is a skew-coninvolutory matrix. This is analogous to the skew-symmetric matrices we've encountered in linear algebra.
\end{remark}
\newpage
\begin{theorem} \label{detbar}
	For a matrix $E \in \Mc{n}$, $det(\bar{E}) = \overline{det(E)}$.
}
\end{theorem}

\begin{proof}
	We prove by mathematical induction. 
	\newline
	\newline \textbf{Base Case}: 
	For $E$ = 
	\begin{pmatrix}
		\bar{a} & \bar{b} \\
		\bar{c} & \bar{d}
	\end{pmatrix}, 
	$det(\bar{E})$ & =
	\begin{vmatrix}
		\bar{a} & \bar{b} \\
		\bar{c} & \bar{d}
	\end{vmatrix} $= \overline{ad} - \overline{bc} = \overline{ad-bc} = \overline{det(E)}$
	\newline
	\newline
	\textbf{Induction Hypothesis}:
	Suppose $det(\bar{E}) = \overline{det(E)}$ holds for $E \in \Mc{n}$.
	\newline
	Let $X \in \Mc{n+1}$. Then, $$det(\overline{X}) = \overline{\sum_{j=1}^{n+1} a_{ij}c_{ij}} = \sum_{j=1}^{n+1} \overline{a_{ij}}\text{ }\overline{c_{ij}}$$ is the $i^{th}$ row expansion of an $(n+1)\times (n+1)$ matrix where $\overline{c_{ij}}$ is the cofactor of $\overline{a_{ij}}$.
	\newline
	Note that $\overline{c_{ij}} = (-1)^{i+j}\overline{M_{ij}}$ where $\overline{M_{ij}}$ is the determinant of the $n\times n$ matrix obtained by deleting the $i^{th}$ row and the $j^{th}$ column of the original matrix.
	\newline
	By I.H., $\overline{M_{ij}}$ is the determinant of an $n\times n$ conjugate matrix. Thus, we see that we are computing for the determinant of an $(n+1)\times (n+1)$ conjugate matrix. 
\end{proof}

\subsection{Skew-Coninvolutory Complex Matrices}

We now show and prove a result concerning whether or not $\mathscr{D}_n(\HH)$ is empty when $n$ is odd as seen in \cite{stamaria}.
\newpage
\begin{theorem}
	$\mathscr{D}_{n}(\HH)$ is empty when $n$ is odd.
\end{theorem}

\begin{proof}
	If $E$ is skew-coninvolutory then $E\bar{E} = -I_n$. \newline Taking the determinant of both sides, 
	\begin{align*}
		det(E\bar{E}) &= det(-I_n) \\
		det(E)det(\bar{E}) &= (-1)^n \\
		det(E)\overline{det(E)} &= (-1)^n \text{, by Theorem \ref{detbar}} \\
		|det(E)|^2 &= (-1)^n
	\end{align*}
	Since $|det(E)|^2 > 0$, $(-1)^n > 0$. Hence, $n$ must be even.
\end{proof}

\section{Quaternion Properties}

\subsection{Addition and Multiplication of Quaternions}

Recall in Chapter 1 - the four-dimensional algebra of quaternions is generated by the basis elements $\{1,\ihat,\jhat,\khat\}$ such that 
\begin{equation} \label{quat_eq}
\ihat^2 = \jhat^2 = \khat^2 = \ihat \jhat \khat = -1
\end{equation}

From the above equation, we can easily derive the following:
\begin{align*}
	\jhat\khat &= \ihat & \khat\jhat &= -\ihat \\
	\khat\ihat &= \jhat & \ihat\khat &= -\jhat \\
	\ihat\jhat &= \khat & \jhat\khat &= -\khat
\end{align*}

Notice that the quaternions are not commutative under multiplication. In general, for quaternions $q_1 = a_1 + b_1\ihat + c_1\jhat + d_1\khat$ and $q_2 = a_2 + b_2\ihat + c_2\jhat + d_2\khat$, 
\begin{align*}
	q_1q_2 &= (a_1a_2 - b_1b_2 - c_1c_2 - d_1d_2) + (a_1b_2 + b_1a_2 + c_1d_2 - d_1c_2)\ihat \\
		   &+ (a_1c_2 - b_1d_2 + c_1a_2 + d_1b_2)\jhat + (a_1d_2 + b_1c_2 - c_1b_2 + d_1a_2)\khat
\end{align*}

Quaternions are, however, commutative under addition where $q_1+q_2 = (a_1+a_2) + (b_1+b_2)\ihat + (c_1+c_2)\jhat + (d_1+d_2)\khat$.

%A quaternion $q = a + b\ihat + c\jhat + d\khat$ can be written as $q = (a+b\ihat) + (c+d\ihat)\jhat = (a+b\ihat) + (d-c\ihat)\khat$ where $a+b\ihat,c+d\ihat,d-c\ihat \in \C$. We can therefore view the set of quaternions as a two-dimensional algebra over \C \cite{stamaria}.
