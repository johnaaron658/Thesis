Ever since their discovery by William Rowan Hamilton in 1843, Quaternions have found extensive use in solving problems both in theoretical and applied mathematics - notably on the problem of 3D rotation. 

Computations regarding 3D rotations use 4x4 matrices with real entries like the ones shown in Figure \ref{4x4}. We call any set of three angles that represent a rotation applied in some order around the principal axes as \emph{Euler Angles} (in this case $\alpha$, $\beta$, and $\gamma$) \cite{lerios}. Computations with these matrices, however, are a bit tedious and require more elementary arithmetic operations \cite{lerios}. It's also more difficult to determine the axis and angle of rotation using Euler angles \cite{lerios}. Furthermore, this method is susceptible to a problem in mechanics known as the \emph{Gimbal Lock} \cite{jia}.

\begin{figure}[h]
	\begin{align*}
			\Rx{R}{x}{\alpha} &=
			\begin{pmatrix}
				1 & 0 & 0 & 0 \\
				0 & \cos{\alpha} & -\sin{\alpha} & 0 \\
				0 & \sin{\alpha} & \cos{\alpha} & 0 \\
				0 & 0 & 0 & 1
			\end{pmatrix}
			&
			\Rx{R}{y}{\beta} &=
			\begin{pmatrix}
				\cos{\beta} & 0 & \sin{\beta} & 0 \\
				0 & 1 & 0 & 0 \\
				-\sin{\beta} & 0 & \cos{\beta} & 0 \\
				0 & 0 & 0 & 1
			\end{pmatrix} \\
			(a) & \text{Rotation by $\alpha$ in the x-axis} 
			& 
			(b) & \text{Rotation by $\beta$ in the y-axis}	
	 \end{align*} 
		 \begin{align*}
			\Rx{R}{z}{\gamma} &=
			\begin{pmatrix}
				\cos{\gamma} & -\sin{\gamma} & 0 & 0 \\
				\sin{\gamma} & \cos{\gamma} &  0 & 0 \\
				0 & 0 & 1 & 0 \\
				0 & 0 & 0 & 1
		 \end{pmatrix} \\
		 (c) & \text{Rotation by $\gamma$ in the z-axis}	
		\end{align*}
		\caption{4x4 Rotation Matrices about the Principal Axes}
		\label{4x4}
\end{figure}

\noindent The gimbal lock is a phenomenon that occurs when two of the moving axes x, y, and z (more commonly known as "pitch", "yaw", and "roll" respectively) coincide - resulting in a loss of one degree of freedom for the object being rotated \cite{jia}. 

%figure explaining the gimbal lock phenomenon

Quaternions do not suffer from the gimbal lock. They are also found to be more compact - requiring less elementary arithmetic operations to perform rotation composition than rotation matrices \cite{lerios}. The axis and angle of rotation can also be easily deduced. Let $\vec{q}$ be the purely imaginary parts of the quaternion $q = a + b\ib + c\jb + d\kb $ ,i.e., $\vec{q} = b\ib + c\jb + d\kb$. It can be shown that $$\frac{\vec{q}}{\sqrt{b^2+c^2+d^2}} \text{ is the axis of rotation and }$$ $$\theta \text{ satisfying } \sin{\theta/2} = \sqrt{b^2+c^2+d^2} \text{ and } \cos{\theta/2} = a \text{ is the angle of rotation. \cite{lerios}}$$

Quaternions are used today in robotics, three-dimensional computer graphics, computer vision, crystallographic texture analysis, navigation, and molecular dynamics. 

Mathematicians have made advancements in developing the theory of Quaternions. Notably, as one of the central points of this topic, we look into the concept of a \emph{Quaternionic Matrix} and the implications it has on certain definitions that were already established in Linear Algebra. 

One such implication is the concept of a determinant in the context of quaternionic matrices. In linear algebra, we saw that we can extend the definition of the determinant to matrices with complex entries \cite{stamaria}. This is possible because the complex numbers are commutative under complex multiplication \cite{aslaksen}. 

Certain problems arise if we attempt to extend the classical definition to the quaternions because quaternions are not commutative under quaternion multiplication \cite{aslaksen}. In \cite{aslaksen}, Aslaksen revisits the properties we associate with determinants and gives three conditions called \emph{axioms} that should be satisfied in order for a definition of a determinant to be valid and useful:
\begin{enumerate}
	\item $det(A) = 0$ if and only if $A$ is singular.
	\item $det(AB) = det(A)det(B)$ for all quaternionic matrices $A$ and $B$.
	\item If $A'$ is obtained by adding a left-multiple of a row to another row or a right-multiple of a column to another column, then $det(A')=det(A)$.
\end{enumerate}

Over the years, several mathematicians have come up with different ways to define a determinant for quaternionic matrices - the Cayley determinant (by Arthur Cayley in 1845), the Study determinant (by Eduard Study in 1920), the Dieudonne determinant, and Moore's determinant. Aslaksen showed whether or not these different definitions satisfy the above conditions \cite{aslaksen}.

We will look at a particular problem that will require the concept of a determinant - that is, to determine whether or not the set of all $n \times n$ \emph{Skew-coninvolutory Quaternionic Matrices} (denoted by $\mathscr{D}_n(\HH)$) is empty when $n$ is odd. 

In \cite{stamaria}, Sta. Maria provided a simple proof to the fact that the set of all $n \times n$ skew-coninvolutory \emph{complex} matrices is empty when $n$ is odd. The method of proof involved using the determinant defined for complex matrices (which is not different from the classical determinant for matrices with real entries).

In this paper, we will discuss the theory behind quaternionic determinants - particularly, the Study determinant. We will then use the Study determinant to extend the result by Sta. Maria to the set of all skew-coninvolutory quaternionic matrices, i.e., we will show that the set of all $n \times n$ skew-coninvolutory quaternionic matrices is empty when $n$ is odd. 

\newpage
\section{List of Symbols}

\begin{itemize}
	\item $\Mr{n}$ - set of all $n\times n$ matrices with real entries.
	\item $\Mc{n}$ - set of all $n\times n$ matrices with complex entries.
	\item $\Mh{n}$ - set of all $n\times n$ matrices with quaternion entries.
	\item $\mathscr{D}_n(\C)$ - set of all $n\times n$ skew-coninvolutory matrices with complex entries.
	\item $\mathscr{D}_n(\HH)$ - set of all $n\times n$ skew-coninvolutory matrices with quaternion entries.
	\item $\cdet{ }$ - determinant of a matrix in $\Mc{n}$.
	\item $\rdet{ }$ - determinant of a matrix in $\Mr{n}$.
\end{itemize}