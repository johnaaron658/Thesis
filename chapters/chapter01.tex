Ever since their discovery by William Rowan Hamilton in 1843, Quaternions have found extensive use in solving problems both in theoretical and applied mathematics - particularly on the problem of 3D rotation. 
Computations regarding the matter use 4x4 matrices with real entries like the ones shown below.

\begin{figure}[h]
	\begin{align*}
			\Rx{R}{x}{\alpha} &=
			\begin{pmatrix}
				1 & 0 & 0 & 0 \\
				0 & \cos{\alpha} & -\sin{\alpha} & 0 \\
				0 & \sin{\alpha} & \cos{\alpha} & 0 \\
				0 & 0 & 0 & 1
			\end{pmatrix}
			&
			\Rx{R}{y}{\beta} &=
			\begin{pmatrix}
				\cos{\beta} & 0 & \sin{\beta} & 0 \\
				0 & 1 & 0 & 0 \\
				-\sin{\beta} & 0 & \cos{\beta} & 0 \\
				0 & 0 & 0 & 1
			\end{pmatrix}	
	 \end{align*} 
		 \begin{equation*}
			\Rx{R}{z}{\gamma} &=
			\begin{pmatrix}
				\cos{\gamma} & -\sin{\gamma} & 0 & 0 \\
				\sin{\gamma} & \cos{\gamma} &  0 & 0 \\
				0 & 0 & 1 & 0 \\
				0 & 0 & 0 & 1
		 \end{pmatrix}	
		\end{equation*}
\end{figure}

Quaternions, however, were found to be considerably more compact - allowing for easy deduction of the axis and angle of rotation from the components \cite{lerios}. Because of this, quaternions are used today in robotics, three-dimensional computer graphics, computer vision, crystallographic texture analysis, navigation, and molecular dynamics. 

Mathematicians have made advancements in developing the theory of Quaternions. Notably, as one of the central points of this topic, we look into the concept of a \emph{Quaternionic Matrix} and the implications it has on certain definitions that were already established in Linear Algebra. 

One such implication is the concept of a determinant in the context of quaternionic matrices. Certain problems arise due to the fact that quaternions are not commutative under quaternion multiplication \cite{aslaksen}. Because of this, we see in \cite{aslaksen} that there are several ways to define a determinant for quaternionic matrices. In \cite{aslaksen}, we encounter three conditions that should be satisfied in order for a definition to be valid, namely:
\begin{enumerate}
	\item $det(A) = 0$ if and only if $A$ is singular.
	\item $det(AB) = det(A)det(B)$ for all quaternionic matrices $A$ and $B$
	\item If $A'$ is obtained by adding a left-multiple of a row to another row or a right-multiple of a column to another column, then $det(A')=det(A)$
\end{enumerate}
%\begin{definition}[Quaternion]
%	The four-dimensional algebra of \emph{Quaternions} is generated by the basis elements $\{1,\ihat,\jhat,\khat\}$ such that $\ihat^2 = \jhat^2 = \khat^2 = \ihat \jhat \khat = -1$. $\HH := \{\quat{a}{b}{c}{d}|a,b,c,d \in \R\}$.
%\end{definition}

