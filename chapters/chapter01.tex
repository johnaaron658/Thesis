Ever since their discovery by William Rowan Hamilton in 1843, Quaternions have found extensive use in solving problems both in theoretical and applied mathematics - particularly on the problem of 3D rotation. Computations regarding the said matter use 4x4 matrices with real entries like the ones shown below.

\begin{figure}[h]
	\begin{align*}
			\Rx{R}{x}{\alpha} &=
			\begin{pmatrix}
				1 & 0 & 0 & 0 \\
				0 & \cos{\alpha} & -\sin{\alpha} & 0 \\
				0 & \sin{\alpha} & \cos{\alpha} & 0 \\
				0 & 0 & 0 & 1
			\end{pmatrix}
			&
			\Rx{R}{y}{\beta} &=
			\begin{pmatrix}
				\cos{\beta} & 0 & \sin{\beta} & 0 \\
				0 & 1 & 0 & 0 \\
				-\sin{\beta} & 0 & \cos{\beta} & 0 \\
				0 & 0 & 0 & 1
			\end{pmatrix}	
	 \end{align*} 
		 \begin{equation*}
			\Rx{R}{z}{\gamma} &=
			\begin{pmatrix}
				\cos{\gamma} & -\sin{\gamma} & 0 & 0 \\
				\sin{\gamma} & \cos{\gamma} &  0 & 0 \\
				0 & 0 & 1 & 0 \\
				0 & 0 & 0 & 1
		 \end{pmatrix}	
	}
\end{equation*}
\end{figure}

Quaternions are used today in robotics, three-dimensional computer graphics, computer vision, crystallographic texture analysis, navigation, and molecular dynamics. 

