\documentclass[12pt]{report}
\usepackage[utf8]{inputenc}
\usepackage[english]{babel}
\usepackage[mathscr]{euscript}
\usepackage{titlesec}
\usepackage{amsmath}
\usepackage{amsthm}
\usepackage{amssymb}
\usepackage{caption}
\usepackage{comment}
\usepackage{subcaption}
\usepackage{graphicx}
\usepackage{epsfig}
\usepackage{upcebumath}  
\usepackage{UPnotations}
\usepackage{hyperref}
\usepackage{rotating}
%\usepackage{float}
%\usepackage{layout}
%\usepackage[hmargin={1.5in,1in},vmargin=1in,includehead,includefoot]{geometry}
\graphicspath{{images/}}

% THEOREM MACROS
%\theoremstyle{definition}
%\newtheorem{definition}{Definition}[section]
%\newtheorem{lemma}{Lemma}[section]
%\theoremstyle{theorem}
%\newtheorem{theorem}{Theorem}[section]
\newtheorem{prop}{Proposition}
\theoremstyle{definition}
\newtheorem{ex}{Example}[section]
%\theoremstyle{remark}
%\newtheorem*{remark}{Remark}
%\renewcommand\qedsymbol{$\blacksquare$}

% MATH SHORTCUTS
\DeclareMathOperator{\R}{\mathbb{R}}
\DeclareMathOperator{\CC}{\mathbb{C}}
\DeclareMathOperator{\HH}{\mathbb{H}}
\newcommand{\Rx}[3]{{#1}_{#2}{(#3)}}
\newcommand{\Mr}[1]{\Rx{$M$}{#1}{\R}}
\newcommand{\Mc}[1]{\Rx{$M$}{#1}{\CC}}
\newcommand{\Mh}[1]{\Rx{$M$}{#1}{\HH}}
\newcommand{\Mf}[1]{\Rx{$M$}{#1}{$F$}}
\newcommand{\nbf}[1]{\textbf{#1}}
\newcommand{\ib}{\nbf{i}}
\newcommand{\jb}{\nbf{j}}
\newcommand{\kb}{\nbf{k}}
\newcommand{\ihat}{\hat{\ib}}
\newcommand{\jhat}{\hat{\jb}}
\newcommand{\khat}{\hat{\kb}}
\newcommand{\quat}[4]{#1 + #2\ib + #3\jb + #4\kb}
\newcommand{\ccdet}{det_{\CC}}
\newcommand{\rrdet}{det_{\R}}
\newcommand{\cdet}[1]{det_{\CC}(#1)}
\newcommand{\rdet}[1]{det_{\R}(#1)}
\newcommand{\genmatk}[4]{
	\begin{matrix}
	 #1 & \ldots & #2 \\
	 \vdots & \ddots & \vdots \\
	 #3 & \ldots & #4
	 \end{matrix}
}
\newcommand{\genmat}[1]{
	\genmatk{#1_{11}}{#1_{1n}}{#1_{n1}}{#1_{nn}}
}
\newcommand{\pgenmatk}[4]{
	\begin{pmatrix}
	 #1 & \ldots & #2 \\
	 \vdots & \ddots & \vdots \\
	 #3 & \ldots & #4
	 \end{pmatrix}
}
\newcommand{\pgenmat}[1]{
	\pgenmatk{#1_{11}}{#1_{1m}}{#1_{m1}}{#1_{mm}}
}

% MATH ENVIRONMENT SETUP
%\setlength\arraycolsep{6pt}
%\renewcommand{\arraystretch}{0.75}
\newenvironment{proof}{\noindent \textit{Proof.}~}{\hfill$\square$}

% DOCUMENT SETUP
%\setlength{\parindent}{4em}
%\setlength{\parskip}{1em}
%\renewcommand{\baselinestretch}{2.0}
%\makeatletter
%\def\@makechapterhead#1{
%  {\parindent \z@ \raggedright \normalfont
%    \ifnum \c@secnumdepth >\m@ne
%        \huge\bfseries \@chapapp\space \thechapter
%        \par\nobreak
%        \vskip 20\p@
%    \fi
%    \interlinepenalty\@M
%    \Huge \bfseries #1\par\nobreak
%    \vskip 40\p@
%  }}
%\def\@makeschapterhead#1{
%  {\parindent \z@ \raggedright
%    \normalfont
%    \interlinepenalty\@M
%    \Huge \bfseries  #1\par\nobreak
%    \vskip 40\p@
%  }}
%\makeatother


% MAIN DOCUMENT
\begin{document}
	% DETAILS
	\iffalse
	\title{An Application of the Study Determinant on Skew-Coninvolutory Quaternionic Matrices}

	\authorfirst{John Aaron Q.}
	\authorlast{Alcoseba}

	\degree{Bachelor of Science in Mathematics}
	\thesistype{Undergraduate Thesis}
	\thesisdocument{Thesis}
	\defensedate{May 29, 2017}
	\submitmonth{June}
	\submityear{2017}
	\adviser{Dr. Lorna S. Almocera}
	\reader{Someone}

	\bothabsenttrue

	\coadviser{co-Adviser, Ph.D.}

	\chairman{Jonnifer R. Sinogaya, Ph.D.} 
	\abs{Type your abstract here.  Your abstract should have a minimum
	of 150 words and a maximum of 300 words.}

	\acknowledge{Type your acknowledgements here. It is customary to acknowledge special assistance from
	extramural agencies. There is no obligation that assistance received from members of the dissertation or
	thesis committee be acknowledged. \medskip

	Acknowledgments should be couched in terms consistent with the
	scholarly nature of the work. Your name and the date should not appear on this page.}

	\beforepreface
	\tableofcontents

	\newpage

	\acknowledgetrue
	\acknowledgepage

	\abstractpage
	%\listoftables
	%\listoffigures

	\afterpreface
	\fi
	%main body
	\chapter{Introduction} 
	\section{Significance of Quaternions}

Ever since their discovery by William Rowan Hamilton in 1843, Quaternions have found extensive use in solving problems both in theoretical and applied mathematics - notably on the problem of 3D rotation. 

\begin{definition}[Quaternion]
	The four-dimensional algebra of \emph{Quaternions} is generated by the basis elements $\{1,\ib,\jb,\kb\}$ such that $\ib^2 = \jb^2 = \kb^2 = \ib \jb \kb = -1$. $\HH := \{\quat{a}{b}{c}{d} | a,b,c,d \in \R\}$.
\end{definition}

Computations regarding 3D rotations use 4x4 matrices with real entries like the ones shown in Figure \ref{4x4}. We call any set of three angles that represent a rotation applied in some order around the principal axes as \emph{Euler Angles} (in this case $\alpha$, $\beta$, and $\gamma$) \cite{lerios}. Computations with these matrices, however, are a bit tedious and require more elementary arithmetic operations \cite{lerios}. It's also more difficult to determine the axis and angle of rotation using Euler angles \cite{lerios}. Furthermore, this method is susceptible to a problem in mechanics known as the \emph{Gimbal Lock} \cite{jia}.

\begin{figure}[h]
	\begin{align*}
			\Rx{R}{x}{\alpha} &=
			\begin{pmatrix}
				1 & 0 & 0 & 0 \\
				0 & \cos{\alpha} & -\sin{\alpha} & 0 \\
				0 & \sin{\alpha} & \cos{\alpha} & 0 \\
				0 & 0 & 0 & 1
			\end{pmatrix}
			&
			\Rx{R}{y}{\beta} &=
			\begin{pmatrix}
				\cos{\beta} & 0 & \sin{\beta} & 0 \\
				0 & 1 & 0 & 0 \\
				-\sin{\beta} & 0 & \cos{\beta} & 0 \\
				0 & 0 & 0 & 1
			\end{pmatrix} \\
			(a) & \text{Rotation by $\alpha$ in the x-axis} 
			& 
			(b) & \text{Rotation by $\beta$ in the y-axis}	
	 \end{align*} 
		 \begin{align*}
			\Rx{R}{z}{\gamma} &=
			\begin{pmatrix}
				\cos{\gamma} & -\sin{\gamma} & 0 & 0 \\
				\sin{\gamma} & \cos{\gamma} &  0 & 0 \\
				0 & 0 & 1 & 0 \\
				0 & 0 & 0 & 1
		 \end{pmatrix} \\
		 (c) & \text{Rotation by $\gamma$ in the z-axis}	
		\end{align*}
		\caption{4x4 Rotation Matrices about the Principal Axes}
		\label{4x4}
\end{figure}

\noindent The gimbal lock is a phenomenon that occurs when two of the moving axes x, y, and z (more commonly known as "pitch", "yaw", and "roll" respectively) coincide - resulting in a loss of one degree of freedom for the object being rotated \cite{jia}. 

%figure explaining the gimbal lock phenomenon

Quaternions do not suffer from the gimbal lock. They are also found to be more compact - requiring less elementary arithmetic operations to perform rotation composition than rotation matrices \cite{lerios}. The axis and angle of rotation can also be easily deduced. Let $\vec{q}$ be the purely imaginary parts of the quaternion $q = a + b\ib + c\jb + d\kb $ ,i.e., $\vec{q} = b\ib + c\jb + d\kb$. It can be shown that $$\frac{\vec{q}}{\sqrt{b^2+c^2+d^2}} \text{ is the axis of rotation and }$$ $$\theta \text{ satisfying } \sin{\theta/2} = \sqrt{b^2+c^2+d^2} \text{ and } \cos{\theta/2} = a \text{ is the angle of rotation. \cite{lerios}}$$

Quaternions are used today in robotics, three-dimensional computer graphics, computer vision, crystallographic texture analysis, navigation, and molecular dynamics. 

\section{Determinants of Quaternionic Matrices}

Mathematicians have made advancements in developing the theory of Quaternions. Notably, as one of the central points of this topic, we look into the concept of a \emph{Quaternionic Matrix} and the implications it has on certain definitions that were already established in Linear Algebra. 

One such implication is the concept of a determinant in the context of quaternionic matrices. In linear algebra, we see that we can extend the definition of the determinant to matrices with complex entries \cite{stamaria}. This is possible because the complex numbers are commutative under complex multiplication \cite{aslaksen}. 

Certain problems arise if we attempt to extend the classical definition to the quaternions because quaternions are not commutative under quaternion multiplication \cite{aslaksen}. \cite{aslaksen} revisits the properties we associate with determinants and gives three conditions called \emph{axioms} that should be satisfied in order for a definition of a determinant to be valid and useful:
\begin{enumerate}
	\item $det(A) = 0$ if and only if $A$ is singular.
	\item $det(AB) = det(A)det(B)$ for all quaternionic matrices $A$ and $B$.
	\item If $A'$ is obtained by adding a left-multiple of a row to another row or a right-multiple of a column to another column, then $det(A')=det(A)$.
\end{enumerate}

Over the years, several mathematicians have come up with different ways to define a determinant for quaternionic matrices - the Cayley determinant (by Arthur Cayley in 1845), the Study determinant (by Eduard Study in 1920), the Dieudonne determinant, and Moore's determinant. \cite{aslaksen} showed whether or not these different definitions satisfy the above conditions. We will be discussing these determinants in greater detail in the following chapter.

\section{Skew-Coninvolutory Quaternionic Matrices}

\cite{stamaria} provided a simple proof to the fact that the set of all $n \times n$ Skew-Coninvolutory Matrices with complex entries (denoted by $\mathscr{D}_n(\C)$) is empty when $n$ is odd. The method of proof involved using the determinant defined for complex matrices (which is not different from the classical determinant for matrices with real entries).
\newpage
In this paper, we attempt to extend this result for quaternionic matrices, i.e., we will investigate whether or not the set of all $n \times n$ Skew-Coninvolutory Matrices with quaternion entries (denoted by $\mathscr{D}_n(\HH)$) is, again, empty when $n$ is odd. Furthermore, we will draw the same method of proof - using the concept of a determinant for quaternionic matrices to obtain the same results.

\section{Symbols}

\begin{itemize}
	\item $\Mr{n}$ - set of all $n\times n$ matrices with real entries.
	\item $\Mc{n}$ - set of all $n\times n$ matrices with complex entries.
	\item $\Mh{n}$ - set of all $n\times n$ matrices with quaternion entries.
	\item $\mathscr{D}_n(\C)$ - set of all $n\times n$ skew-coninvolutory matrices with complex entries.
	\item $\mathscr{D}_n(\HH)$ - set of all $n\times n$ skew-coninvolutory matrices with quaternion entries.
\end{itemize}

	\chapter{Preliminaries}
	\section{Complex Matrices}

\begin{definition}[Conjugate Matrix]
	A \emph{conjugate matrix} is a matrix $\bar{E}$ obtained from $E$ by taking the complex conjugate of every entry of $E$.
\end{definition}

\begin{definition}[Coninvolutory Matrix]
	A matrix is said to be \emph{coninvolutory} if $E\bar{E} = I_n$ for $E \in \Mc{n}$.
\end{definition}

\begin{remark}
	By manipulation, we obtain $E^{-1} = \bar{E}$. Hence, we may also say that a matrix whose inverse is its own conjugate matrix is a coninvolutory matrix. Furthermore, we see that coninvolutory matrices are the extension of complex numbers with modulus 1 \cite{stamaria}.
\end{remark}

\begin{definition}[Skew-Coninvolutory Matrix]
	A matrix is said to be \emph{skew-coninvolutory} if $E\bar{E} = -I_n$ for $E \in \Mc{n}$.
\end{definition}

\begin{remark}
	Again, we may say that a matrix whose inverse is the negative of its own conjugate matrix is a skew-coninvolutory matrix. This is analogous to the skew-symmetric matrices we've encountered in linear algebra.
\end{remark}

\begin{theorem} \label{detbar}
	For a matrix $E \in \Mc{n}$, $det(\bar{E}) = \overline{det(E)}$.
}
\end{theorem}

\begin{proof}
	We prove by mathematical induction. 
	\newline
	\newline \textbf{Base Case}: 
	For $E$ = 
	\begin{pmatrix}
		\bar{a} & \bar{b} \\
		\bar{c} & \bar{d}
	\end{pmatrix}, 
	$det(\bar{E})$ & =
	\begin{vmatrix}
		\bar{a} & \bar{b} \\
		\bar{c} & \bar{d}
	\end{vmatrix} $= \overline{ad} - \overline{bc} = \overline{ad-bc} = \overline{det(E)}$
	\newline
	\newline
	\textbf{Induction Hypothesis}:
	Suppose $det(\bar{E}) = \overline{det(E)}$ holds for $E \in \Mc{n}$.
	\newline
	Let $X \in \Mc{n+1}$. Then, $$det(\overline{X}) = \overline{\sum_{j=1}^{n+1} a_{ij}c_{ij}} = \sum_{j=1}^{n+1} \overline{a_{ij}}\text{ }\overline{c_{ij}}$$ is the $i^{th}$ row expansion of an $(n+1)\times (n+1)$ matrix where $\overline{c_{ij}}$ is the cofactor of $\overline{a_{ij}}$.
	\newline
	Note that $\overline{c_{ij}} = (-1)^{i+j}\overline{M_{ij}}$ where $\overline{M_{ij}}$ is the determinant of the $n\times n$ matrix obtained by deleting the $i^{th}$ row and the $j^{th}$ column of the original matrix.
	\newline
	By I.H., $\overline{M_{ij}}$ is the determinant of an $n\times n$ conjugate matrix. Thus, we see that we are computing for the determinant of an $(n+1)\times (n+1)$ conjugate matrix. 
\end{proof}

\subsection{Skew-Coninvolutory Complex Matrices}

We now show and prove a result concerning whether or not $\mathscr{D}_n(\C)$ is empty when $n$ is odd as seen in \cite{stamaria}.

\begin{theorem}
	$\mathscr{D}_{n}(\C)$ is empty when $n$ is odd.
\end{theorem}

\begin{proof}
	If $E \in \mathscr{D}_{n}(\C)$ then $E\bar{E} = -I_n$. \newline Taking the determinant of both sides, 
	\begin{align*}
		det(E\bar{E}) &= det(-I_n) \\
		det(E)det(\bar{E}) &= (-1)^n \\
		det(E)\overline{det(E)} &= (-1)^n \text{, by Theorem \ref{detbar}} \\
		|det(E)|^2 &= (-1)^n
	\end{align*}
	Since $|det(E)|^2 > 0$, $(-1)^n > 0$. Hence, $n$ must be even.
\end{proof}

\section{Quaternion Basics}

\subsection{Multiplication and Addition}

Recall in Chapter 1 - the four-dimensional algebra of quaternions is generated by the basis elements $\{1,\ib,\jb,\kb\}$ such that 
\begin{equation} \label{quat_eq}
\ib^2 = \jb^2 = \kb^2 = \ib \jb \kb = -1
\end{equation}

From the above equation, we can easily derive the following:
\begin{align*}
	\jb\kb &= \ib & \kb\jb &= -\ib \\
	\kb\ib &= \jb & \ib\kb &= -\jb \\
	\ib\jb &= \kb & \jb\kb &= -\kb
\end{align*}

Notice that the quaternions are not commutative under \emph{multiplication}. In general, for quaternions $q_1 = a_1 + b_1\ib + c_1\jb + d_1\kb$ and $q_2 = a_2 + b_2\ib + c_2\jb + d_2\kb$, 
\begin{align*}
	q_1q_2 &= (a_1a_2 - b_1b_2 - c_1c_2 - d_1d_2) + (a_1b_2 + b_1a_2 + c_1d_2 - d_1c_2)\ib \\
		   &+ (a_1c_2 - b_1d_2 + c_1a_2 + d_1b_2)\jb + (a_1d_2 + b_1c_2 - c_1b_2 + d_1a_2)\kb
\end{align*}

%insert vector representation

Quaternions are, however, commutative under \emph{addition} where $q_1+q_2 = (a_1+a_2) + (b_1+b_2)\ib + (c_1+c_2)\jb + (d_1+d_2)\kb$.

\subsection{Other Operations and Properties}

\begin{definition}[$\HH$-Conjugate]
	The $\HH$-Conjugate of a quaternion $q = \quat{a}{b}{c}{d}$ is $\bar{q} = a - b\ib - c\jb - d\kb$.
\end{definition}

\begin{remark}
	Notice that $q\bar{q} = (\quat{a}{b}{c}{d})(a - b\ib -c\jb-d\kb) = a^2+b^2+c^2+d^2$.
\end{remark}

\begin{definition}[$\HH$-Norm]
	The $\HH$-Norm of a quaternion $q = \quat{a}{b}{c}{d}$ is $|q| = \sqrt{q\bar{q}} = \sqrt{a^2+b^2+c^2+d^2}$
\end{definition}

\begin{definition}[Inverse]
	The inverse of a quaternion $q$ is $q^{-1}$ such that $q^{-1}q = qq^{-1} = 1$.
\end{definition}

\begin{theorem}
For $q, p, r \in \HH$,
	\begin{enumerate}
		\item $|q|^2 = q\bar{q}$.
		\item If $q\neq 0$, then $q^{-1} = \bar{q}/|q|^2$.
		\item $\overline{qp} = \bar{p}\bar{q}$.
		\item $(qp)^{-1} = p^{-1}q^{-1}$ provided that the inverses of $p$ and $q$ exist.
		\item $(qp)r = q(pr)$ that is, quaternion multiplication is associative.
	\end{enumerate}
\end{theorem}

%insert proof here.

\begin{remark}
	Notice that most of the properties we see in quaternions are merely extensions of the properties we see in complex numbers. 
\end{remark}

\subsection{Quaternionic Matrices}

Most of the definitions we've already mentioned for complex matrices can also be extended in the context of quaternionic matrices.

\begin{definition}[Conjugate Quaternionic Matrix]
	A \emph{conjugate quaternionic matrix} is a matrix $\bar{E}$ obtained from $E$ by taking the $\HH$-conjugate of every entry of $E$.
\end{definition}

\begin{definition}[Skew-Coninvolutory Quaternionic Matrix]
	A quaternionic matrix $E$ is said to be \emph{Skew-Coninvolutory} if $E\bar{E} = -I_n$.
\end{definition}

\section{Matrix Homomorphisms}

We look into functions that make it possible for us to represent complex numbers and quaternions as matrices. These functions are of extreme importance as they are used to define some of the quaternionic determinants we will encounter.

\subsection{Representing Complex Numbers as Real Matrices}

In abstract algebra, we saw that we can define an isomorphism from the field of complex numbers to the 2D-plane ($\R^2$) - a mapping $\Theta : \C \rightarrow \R^2$ where a complex number $a+b\ib$ is mapped to a vector/point $(a,b)$ in the 2D-plane. Therefore, in order to represent complex numbers as real matrices, we have to find a way to view them as linear transformations over $\R^2$. 

Consider the complex function $f(z) = (a+b\ib)z$. We see that the images of $1$ and $\ib$ are $a+b\ib$ and $-b+a\ib$ respectively. Under the isomorphism $\Theta$ (in which case $1$ is mapped to $(1,0)$ while $\ib$ is mapped to $(0,1)$), we seek a matrix in $\Mr{2}$ that maps $(1,0)$ to $\Theta(a+b\ib) = (a,b)$ and $(0,1)$ to $\Theta(-b+a\ib) = (-b,a)$. 
\\
\noindent Let this matrix be $F$ = \begin{pmatrix} \alpha & \beta \\ \chi & \delta \end{pmatrix} where $\alpha, \beta, \chi, \text{ and } \delta \in \R$. Then, 
\begin{equation*}
	\begin{pmatrix} 
		\alpha & \beta \\ 
		\chi & \delta 
	\end{pmatrix} 
	\begin{pmatrix} 
		1 \\ 0 
	\end{pmatrix} = 
	\begin{pmatrix} a \\ b \end{pmatrix} \implies
	\begin{pmatrix}
		\alpha \\ \chi
	\end{pmatrix} =
	\begin{pmatrix} a \\ b \end{pmatrix} \implies 
	\alpha = a; \chi = b \text{ and}

	\begin{pmatrix} 
		\alpha & \beta \\ 
		\chi & \delta 
	\end{pmatrix} 
	\begin{pmatrix} 
		0 \\ 1 
	\end{pmatrix} = 
	\begin{pmatrix} -b \\ a \end{pmatrix} \implies
	\begin{pmatrix}
		\beta \\ \delta
	\end{pmatrix} =
	\begin{pmatrix} -b \\ a \end{pmatrix} \implies
	\beta = -b; \delta = a \\
\end{equation*}
\\
\noindent Therefore, $F$ = \begin{pmatrix} \alpha & \beta \\ \chi & \delta \end{pmatrix} = \begin{pmatrix} a & -b \\ b & a \end{pmatrix}. The matrix $F$ can be seen as the matrix representation of the function $f$ which is defined by multiplying a complex number $z$ by $a+b\ib$. We can therefore see the matrix $F$ as the real matrix representation of the complex number $a+b\ib$.

\begin{remark}
	Notice that the column vectors of the matrix $F$ are where the vectors $(1,0)$ and $(0,1)$ are mapped to. This pattern shows up in most of the matrices that we will be dealing with - the column vectors of a matrix are the images of the basis vectors under the linear transformation.
\end{remark}

\subsection{Homomorphisms from $\Mc{n}$ to $\Mr{2n}$}

In the previous subsection, we saw that we can represent complex numbers as $2\times 2$ real matrices. We can then define a mapping from $\C$ to $\Mr{2}$. We can also show that this mapping is a homomorphism.

\begin{theorem}
	Let $\phi : \C \rightarrow \Mr{2}$ such that $a+b\ib \mapsto$ \begin{pmatrix} a & -b \\ b & a \end{pmatrix}. Then $\phi$ is a homomorphism from $\C$ to $\Mr{2}.
\end{theorem}

\begin{proof}
	Let $z_1 = a+b\ib$ and $z_2 = c+d\ib \in \C$ . \\
	Then $\phi(z_1z_2) = \phi[(a+b\ib)(c+d\ib)] = \phi[(ac-bd)+(ad+bc)\ib] = $\begin{pmatrix} (ac-bd) & -(ad+bc) \\ (ad+bc) & (ac-bd)$ \end{pmatrix}. \\ \\
	Now, $\phi(z_1)\phi(z_2) = \phi[(a+b\ib)]\phi[(c+d\ib)] = $\begin{pmatrix} a & -b \\ b & a \end{pmatrix}\begin{pmatrix} c & -d \\ d & c \end{pmatrix}$ = $\begin{pmatrix} (ac-bd) & -(ad+bc) \\ (ad+bc) & (ac-bd)$ \end{pmatrix}. \\
	Hence, $\phi(z_1z_2) = \phi(z_1)\phi(z_2)$.
\end{proof}

\subsection{Representing Quaternions as Complex Matrices}

%exposition

\subsection{Homomorphisms from $\Mh{n}$ to $\Mc{2n}$}


\section{Quaternionic Determinants}

\subsection{Aslaksen's Axioms and the Cayley Determinant}

\subsection{The Study Determinant}

\subsection{The Dieudonne Determinant}

\subsection{Moore's Determinant}

%A quaternion $q = a + b\ib + c\jb + d\kb$ can be written as $q = (a+b\ib) + (c+d\ib)\jb = (a+b\ib) + (d-c\ib)\kb$ where $a+b\ib,c+d\ib,d-c\ib \in \C$. We can therefore view the set of quaternions as a two-dimensional algebra over \C \cite{stamaria}.


	\chapter{Results and Discussion}
	\section{The Cayley Determinant and Aslaksen's Axioms}

In 1845, 2 years after William Rowan Hamilton discovered quaternions, Arthur Cayley attempted to define the determinant of a quaternionic matrix using the usual formula (we denote the Cayley determinant by $Cdet$), i.e., for a $2 \times 2$ quaternionic matrix $A = $ \begin{pmatrix} a & b \\ c & d \end{pmatrix}, $Cdet(A) = ad - cb$ for $a,b,c,d \in \HH$ \cite{aslaksen}. The same goes for $3 \times 3$ matrices and so on. Taking into account the fact that the quaternions are non-commutative (and the implications it has on linear mappings as will be discussed later), we might ask whether or not this determinant behaves the way we expect - Will it really determine whether or not a quaternionic matrix is singular or not? Will the properties of the determinant still hold? Will the determinant still be a map from $M_{n}(G) \rightarrow G$ (in this case, $G = \HH$)? The last question comes from the fact that the determinants of complex matrices is a map from $\Mc{n} \rightarrow \CC$.  


\subsection{The Determinant Function}

We take a step back and revisit what it means for a mapping to be a determinant. We present the determinant function as defined by J.L. Brenner.

\begin{definition}[Determinant Function]
	For a field $F$, a determinant over the matrices of $\Mf{n}$ is a function $det$ from $\Mf{n}$ into $F$ such that 
	\begin{equation}
	det(AB) = det(A)det(B) = det(B)det(A)
	\end{equation} 
	holds either \textbf{(1)} $\forall A, B \in \Mf{n}$ or \textbf{(2)} $\forall$ invertible $A, B \in \Mf{n}$. 
\end{definition}

Notice that the definition requires the images of $A$ and $B$ to commute (this is not possible for skew-fields like the quaternions). We can see this matter viewed in a more rigorous manner (also in a manner more specific to the quaternions) while discussing Aslaksen's axioms. 

We see that if $det$ is a constant function that only maps to either 0 or 1 (with 0 for singular matrices and 1 for invertible matrices), then $det$ satisfies the above definition \cite{brenner}. The following theorem by Brenner shows that this holds for non-trivial determinants as well and that conditions (1) and (2) are essentially equivalent \cite{brenner}. 

\begin{theorem}
If $det$ is not constantly equal to 1 or 0 (i.e., $det$ is not a mapping $det: \Mf{n} \rightarrow F$ where $F$ is a field with two elements), then $det(B) = 0$ for all singular matrices. 
\end{theorem}
\iffalse
\begin{proof}
	Let $O$ be the zero matrix, $I$ the identity matrix, and $A$ a matrix whose determinant (image under $det$) is neither 1 nor 0. 
	Since $OA = O$, by Axiom 2, $det(O)det(A) = det(O)$. Suppose $det(O) \neq 0$, then $det(A) = 1$ which is a contradiction. Hence, $det(O) = 0$.
	Similarly, since $IA = A$, by Axiom 2, $det(I)det(A) = det(A) \implies det(I) = 1$.
	Also note that for a permutation matrix (an elementary matrix that permutes the rows/columns) $E$, $E^m = I$ for some $m$, hence $det(E) \neq 0$.
	Now, define a matrix
	\begin{equation*}
		D = 
		\begin{pmatrix}
			1 &  &  &  &  &  \\
			 & \ddots & & & & \\
			  & & 1 & & & \\
			  & & & 0 & & \\
			  & & & & \ddots & \\
			  & & & & & 0
		\end{pmatrix}
	\end{equation*}
	Notice that if $D$ is singular, then $det(D) = 0$.
\end{proof}
\fi

\subsection{Aslaksen's Axioms}

In the \emph{Mathematical Intelligencer}, Helmer Aslaksen presented 3 determinant \emph{axioms} which a determinant definition must satisfy in order for it to behave the way we expect, i.e., it has the properties we associate with determinants. These axioms were already introduced in Chapter 1 and we will be discussing them in greater detail here. 
\begin{itemize}
	\item \textbf{Axiom 1.} $det(A) = 0$ if and only if $A$ is singular.
	\item \textbf{Axiom 2.} $det(AB) = det(A)det(B)$ for all quaternionic matrices $A$ and $B$.
	\item \textbf{Axiom 3.} If $A'$ is obtained by adding a left-multiple of a row to another row or a right-multiple of a column to another column, then $det(A')=det(A)$.
\end{itemize}



\begin{lemma}

\end{lemma}

\begin{lemma}

\end{lemma}

\begin{theorem}

\end{theorem}

\section{The Study Determinant}

It was not until 1920, that a new approach in defining a quaternionic determinant was presented in a paper by Eduard Study \cite{aslaksen}. His idea involved transforming quaternionic matrices into complex matrices from which one could then just simply take the determinant \cite{aslaksen}. The method involves homomorphisms between quaternionic, complex, and real matrices.

\subsection{Matrix Homomorphisms}

We look into functions that make it possible for us to represent complex numbers and quaternions as matrices. 
\iffalse
\subsubsection{Representing Complex Numbers as Real Matrices}

In abstract algebra, we saw that we can define a bijection from the field of complex numbers to the 2D-plane ($\R^2$) - a mapping $\Theta : \CC \rightarrow \R^2$ where a complex number $a+b\ib$ is mapped to a vector/point $(a,b)$ in the 2D-plane. Therefore, in order to represent complex numbers as real matrices, we have to find a way to view them as linear transformations over $\R^2$. 

Consider the complex function $f(z) = (a+b\ib)z$. We see that the images of $1$ and $\ib$ are $a+b\ib$ and $-b+a\ib$ respectively. Under the function $\Theta$ (in which case $1$ is mapped to $(1,0)$ while $\ib$ is mapped to $(0,1)$), we seek a matrix in $\Mr{2}$ that maps $(1,0)$ to $\Theta(a+b\ib) = (a,b)$ and $(0,1)$ to $\Theta(-b+a\ib) = (-b,a)$. 
\\
\noindent Let this matrix be $F$ = \begin{pmatrix} \alpha & \beta \\ \chi & \delta \end{pmatrix} where $\alpha, \beta, \chi, \text{ and } \delta \in \R$. Then, 
\begin{equation*}
	\begin{pmatrix} 
		\alpha & \beta \\ 
		\chi & \delta 
	\end{pmatrix} 
	\begin{pmatrix} 
		1 \\ 0 
	\end{pmatrix} = 
	\begin{pmatrix} a \\ b \end{pmatrix} \implies
	\begin{pmatrix}
		\alpha \\ \chi
	\end{pmatrix} =
	\begin{pmatrix} a \\ b \end{pmatrix} \implies 
	\alpha = a; \chi = b \text{ and}

	\begin{pmatrix} 
		\alpha & \beta \\ 
		\chi & \delta 
	\end{pmatrix} 
	\begin{pmatrix} 
		0 \\ 1 
	\end{pmatrix} = 
	\begin{pmatrix} -b \\ a \end{pmatrix} \implies
	\begin{pmatrix}
		\beta \\ \delta
	\end{pmatrix} =
	\begin{pmatrix} -b \\ a \end{pmatrix} \implies
	\beta = -b; \delta = a \\
\end{equation*}
\\
\noindent Therefore, $F$ = \begin{pmatrix} \alpha & \beta \\ \chi & \delta \end{pmatrix} = \begin{pmatrix} a & -b \\ b & a \end{pmatrix}. The matrix $F$ can be seen as the matrix representation of the function $f$ which is defined by multiplying a complex number $z$ by $a+b\ib$. We can therefore see the matrix $F$ as the real matrix representation of the complex number $a+b\ib$.
\fi
\subsubsection{Homomorphisms from $\Mc{n}$ to $\Mr{2n}$}
\iffalse
In the previous subsection, we saw that we can represent complex numbers as $2\times 2$ real matrices. We can then define a mapping from $\CC$ to $\Mr{2}$. We can also show that this mapping is a homomorphism.

\begin{theorem}
	Let $\phi : \CC \rightarrow \Mr{2}$ such that $a+b\ib \mapsto$ \begin{pmatrix} a & -b \\ b & a \end{pmatrix}. Then $\phi$ is an injective homomorphism from $\CC$ to $\Mr{2}.
\end{theorem}

%\begin{proof}
%	Let $z_1 = a+b\ib$ and $z_2 = c+d\ib \in \CC$ . \\
%	Then $\phi(z_1z_2) = \phi[(a+b\ib)(c+d\ib)] = \phi[(ac-bd)+(ad+bc)\ib] = $\begin{pmatrix} (ac-bd) & -(ad+bc) \\ (ad+bc) & (ac-bd)$ \end{pmatrix}. \\ \\
%	Now, $\phi(z_1)\phi(z_2) = \phi[(a+b\ib)]\phi[(c+d\ib)] = $\begin{pmatrix} a & -b \\ b & a \end{pmatrix}\begin{pmatrix} c & -d \\ d & c \end{pmatrix}$ = $\begin{pmatrix} (ac-bd) & -(ad+bc) \\ (ad+bc) & (ac-bd)$ \end{pmatrix}. \\
%	Hence, $\phi(z_1z_2) = \phi(z_1)\phi(z_2)$.
%\end{proof}

\begin{remark}
	We will not include the proof for this theorem as this is merely a special case of Theorem \ref{phimorph} (when $n = 1$). 
\end{remark}
\fi

In order to represent complex matrices as real matrices, notice that every complex matrix can be represented as the sum of a real matrix and a purely imaginary matrix, i.e., for an $n\times n$ matrix $Z$, $Z = A + B\ib$ where $A,B \in \Mr{n}$ \cite{aslaksen}. We define a mapping \begin{equation*} \phi(A+B\ib) = \begin{pmatrix} A & -B \\ B & A \end{pmatrix}  \cite{aslaksen}\end{equation*} 

Before we show that this mapping is an injective homomorphism, we first show that the left distributive laws hold for matrices in $\Mc{n}$.

\begin{theorem}\label{distributive}
	For matrices $A,B,C \in \Mc{n}$, $A(B+C) = AB + AC$.
}
\end{theorem}

\begin{proof}
	Let $A = [a_{ij}]$, $B = [b_{ij}]$, $C = [c_{ij}] \in \Mc{n}$. Then $B+C = [b_{ij}+c_{ij}]$ and \begin{equation} 
	\begin{align*} 
	A(B+C) &= [\sum_{k=1}^{n}a_{ik}(b_{kj}+c_{kj})] = [\sum_{k=1}^{n}(a_{ik}b_{kj}+a_{ik}c_{kj})] \\ 
	&= [\sum_{k=1}^{n}a_{ik}b_{kj} + \sum_{k=1}^{n}a_{ik}c_{kj}] = [\sum_{k=1}^{n}a_{ik}b_{kj}] + [\sum_{k=1}^{n}a_{ik}c_{kj}] = AB + AC 
	\end{align*} \end{equation}
\end{proof}

\begin{remark}
	The same method of proof can be used for the right distributive law. Furthermore, this also holds for matrices in $\Mr{n}$ and $\Mh{n}$.
\end{remark}

\begin{theorem} \label{phimorph}
	Let $\phi: \Mc{n} \rightarrow \Mr{2n}$ such that $C+D\ib \mapsto $ \begin{pmatrix} C & -D \\ D & C \end{pmatrix} where $C+D\ib \in \Mc{n}$. Then $\phi$ is an injective homomorphism. 
\end{theorem}

\begin{proof}
	\textbf{\newline1-1:}
	\begin{equation*}
		\phi(A+B\ib) = \phi(C+D\ib)
		&\implies \begin{pmatrix}A & -B \\ B & A \end{pmatrix} = \begin{pmatrix}C & -D \\ D & C \end{pmatrix}
	\end{equation*}
	\begin{align*}
		\implies A = C \text{ and } B = D \text{ by Matrix Equality}
		\implies A+B\ib = C+D\ib
		\implies \phi \text{ is injective.}
	\end{align*}
	\textbf{Homomorphism: \newline}
	Let $A+B\ib$, $C+D\ib \in \Mc{n}$. Then \begin{align*}
		\phi[(A+B\ib)(C+D\ib)] &= \phi[(A+B\ib)C+(A+B\ib)D\ib] \text{ by Theorem \ref{distributive}} \\
		&= \phi[AC+BC\ib+AD\ib-BD] = \phi[(AC-BD)+(BC+AD)\ib] \\
		&= \begin{pmatrix} (AC-BD) & -(BC+AD) \\ (BC+AD) & (AC-BD) \end{pmatrix}
	\end{align*}
		$\phi[(A+B\ib)]\phi[(C+D\ib)]$ = \begin{pmatrix}A & -B \\ B & A \end{pmatrix}\begin{pmatrix}C & -D \\ D & C \end{pmatrix} \newline \\ = \begin{pmatrix} \genmat{a} & \genmat{-b} \\ \genmat{b} & \genmat{a} \end{pmatrix}\begin{pmatrix} \genmat{c} & \genmat{-d} \\ \genmat{d} & \genmat{c} \end{pmatrix} \\  \\
	= \begin{pmatrix} 
	\genmatk{\sum_{k=1}^{n}a_{1k}c_{k1} - \sum_{k=1}^{n}b_{1k}d_{k1}}{-\sum_{k=1}^{n}a_{1k}d_{kn} - \sum_{k=1}^{n}b_{1k}c_{kn}}{\sum_{k=1}^{n}b_{nk}c_{k1} + \sum_{k=1}^{n}a_{nk}d_{k1}}{-\sum_{k=1}^{n}b_{nk}d_{kn} + \sum_{k=1}^{n}a_{nk}c_{kn}} 
	%\genmatk{\sum_{k=1}^{n}a_{1k}c_{k1} - \sum{k=1}^{n}b_{1k}d_{k1}}{\sum_{k=1}^{n}a_{1k}c_{kn} - \sum{k=1}^{n}b_{1k}d_{kn}}{\sum_{k=1}^{n}a_{nk}c_{k1} - \sum{k=1}^{n}b_{nk}d_{k1}}{\sum_{k=1}^{n}a_{nk}c_{kn} - \sum{k=1}^{n}b_{nk}d_{kn}} &
	%\genmatk{-(\sum_{k=1}^{n}a_{1k}d_{k1} + \sum_{k=1}^{n}b_{1k}c{k1})}{-(\sum_{k=1}^{n}a_{1k}d_{kn} + \sum_{k=1}^{n}b_{1k}c_{kn})}{-(\sum_{k=1}^{n}a_{nk}d_{k1} + \sum_{k=1}^{n}b_{nk}c{k1})}{-(\sum_{k=1}^{n}a_{nk}d_{kn} + \sum_{k=1}^{n}b_{nk}c_{kn})} \\
	%\genmatk{\sum_{k=1}^{n}a_{1k}d_{k1} + \sum_{k=1}^{n}b_{1k}c{k1}}{\sum_{k=1}^{n}a_{1k}d_{kn} + \sum_{k=1}^{n}b_{1k}c_{kn}}{\sum_{k=1}^{n}a_{nk}d_{k1} + \sum_{k=1}^{n}b_{nk}c{k1}}{\sum_{k=1}^{n}a_{nk}d_{kn} + \sum_{k=1}^{n}b_{nk}c_{kn}} &
	%\genmatk{\sum_{k=1}^{n}a_{1k}c_{k1} - \sum{k=1}^{n}b_{1k}d_{k1}}{\sum_{k=1}^{n}a_{1k}c_{kn} - \sum{k=1}^{n}b_{1k}d_{kn}}{\sum_{k=1}^{n}a_{nk}c_{k1} - \sum{k=1}^{n}b_{nk}d_{k1}}{\sum_{k=1}^{n}a_{nk}c_{kn} - \sum{k=1}^{n}b_{nk}d_{kn}}
	\end{pmatrix} \\ \\
	= \begin{pmatrix} (AC-BD) & -(BC+AD) \\ (BC+AD) & (AC-BD) \end{pmatrix}
\end{proof}
\newline
\newline
\begin{definition}[Complex Structure]
	A \emph{complex structure} of a vector space $V$ is defined by the linear map (linear transformation) $J: V \rightarrow V$ such that $J^2 = -I$, where $I$ is the identity map. \cite{wolfram} 
\end{definition}

Complex structures are, in general, linear maps that exhibit the property of the imaginary number $i$, that is, $i^2 = -1$. It is important to note that a linear map \emph{must} commute with scalar multiplication, and thus, representing complex linear maps as real linear maps requires the latter to commute with a complex structure of its vector space (this applies to any associated linear maps between different vector spaces) \cite{aslaksen} \cite{stack}.

We define a matrix \begin{equation*} J = \begin{pmatrix} 0 & -I_n \\ I_n & 0 \end{pmatrix} \end{equation*}. Notice that the matrix $J$ is the image of $iI \in \Mc{n}$ under $\phi$ \cite{aslaksen}. It can be easily shown that $J^2 = -I$. It is obvious that J gives a \emph{complex structure} in $\R^{2n}$. Hence, $\phi(\Mc{n}) = \{P \in \Mr{2n} | JP = PJ\}$, i.e., the real matrix representations of complex matrices are the linear maps in $\Mr{2n}$ that commute with the complex structure \cite{aslaksen}. 

%We will extend the definition of $\phi$ to hold for complex matrices in general. To do this, notice that every complex matrix can be represented as the sum of a real matrix and a purely imaginary matrix, i.e., for an $n\times n$ matrix $N$, $N = C + D\ib$ where $C,D \in \Mr{n}$ \cite{aslaksen}. We see that we can intuitively extend the definition of $\phi$ by defining \begin{equation*} \phi(C+D\ib) = \begin{pmatrix} C & -D \\ D & C \end{pmatrix}  \cite{aslaksen}\end{equation*} 


\iffalse
\subsection{Representing Quaternions as Complex Matrices}

A quaternion $q = a + b\ib + c\jb + d\kb$ can be written as $q = (a+b\ib) + (c+d\ib)\jb = (a+b\ib)+\jb(c-d\ib) = (a+b\ib)-\jb(-c+d\ib)$ where $a+b\ib,c+d\ib,c-d\ib, -c+d\ib \in \CC$. Note that from the latter, we can easily deduce that $v\jb = \jb\bar{v}$ for $v \in \CC$. 

We can therefore view the set of quaternions as a two-dimensional algebra over $\CC$ \cite{stamaria}, i.e., we can define a mapping $\Omega : \HH \rightarrow \CC^2$ such that $\quat{a}{b}{c}{d} \mapsto (a+b\ib,-c+d\ib)$. We can easily show that $\Omega$ is a bijection.

Let us consider the quaternionic function $g(q) = (\quat{a}{b}{c}{d})q$. 


\newcommand{\kmat}{\begin{pmatrix} \kappa & \lambda \\ \mu & \nu \end{pmatrix}}

\newcommand{\vectC}[2]{\begin{pmatrix} #1 \\ #2 \end{pmatrix}}

Let this matrix be $G = $ \kmat, where $\kappa, \lambda, \mu, $ and $\nu \in \CC$. Then, 

\kmat \vectC{1}{0} $=$ \vectC{\kappa}{\mu} $=$ \vectC{a+b\ib}{c+d\ib} $\implies$ $\kappa = a+b\ib$ and $\mu = c+d\ib$.

\kmat \vectC{\ib}{0} $=$ \vectC{\kappa \ib}{\mu \ib} $=$ \vectC{-b+a\ib}{d-c\ib} $\implies$ $\kappa = a+\ib$ and $\mu = $
\fi
\subsubsection{Homomorphisms from $\Mh{n}$ to $\Mc{2n}$}

To represent quaternionic matrices as complex matrices, notice that every quaternionic matrix can be represented as the sum $Y = C + \jb D$ where $C, D \in \Mc{n}$ \cite{aslaksen}. We define a mapping \begin{equation*} \psi(C+\jb D) = \begin{pmatrix} C & -\overline{D} \\ D & \overline{C} \end{pmatrix}  \cite{aslaksen}\end{equation*} 

We can show that $\psi$ is an injective homomorphism using the same proof outline in the previous subsection \cite{aslaksen}.

The non-commutativity of quaternions presents some problems in representing \emph{quaternionic linear maps} as complex linear maps. If we consider a quaternionic linear map say $L(v) = Av$ for $A \in \Mh{n}$ where we take in quaternions as scalars, then, $cAv = c L(v) = L(cv) = Acv$ which is false (considering the base case for $1\times 1$ matrices) \cite{stack}. However, $Avc = L(v)c = L(vc) = Avc$. Hence, we now see that any quaternionic linear map commutes with right scalar multiplication by a quaternion which itself is not a linear map in $\HH$ (in order for it to be a linear map, it in turn, has to commute with other quaternions)\cite{stack} \cite{aslaksen}. This poses a problem because it implies that there is no matrix representation for right scalar multiplication \cite{aslaksen}. However, in \cite{aslaksen}, we see that we can consider a linear map $\widetilde{R_j}$ in $\CC^{2n}$ as the image of right scalar multiplication by $\jb$ under the homomorphism. $\widetilde{R_j}$ corresponds to multiplying $v \in \CC^{2n}$ by the matrix $J$ and then conjugating \cite{aslaksen}. This gives a quaternionic structure in $\CC^{2n}$ and thus, a quaternionic linear map corresponds to a complex linear map $Q$ that commutes with $\widetilde{R_j}$, i.e., $Q \overline{Jv} = \overline{JQv}$ for $v \in \CC^{2n}$. It can be easily shown that the latter holds if and only if $\overline{Q}J = JQ$ using the fact that $Q\overline{Jv} = \overline{QJv}$. Thus, $\psi(\Mh{n}) = \{Q \in \Mc{2n} |\overline{Q}J = JQ\}.

\subsection{Study Determinant}

\begin{definition}
	The Study Determinant is defined by $Sdet M = \cdet{\psi{M}} = \sqrt{\rdet{\phi(\psi(M))}}.
\end{definition}

It can be shown that the Study Determinant satisfies all of Aslaksen's axioms \cite{aslaksen}.

\section{Main Result}
\begin{prop}
 $\mathscr{D}_n(\HH)$ is empty when $n$ is odd.
\end{prop}
\begin{proof}
	If $F \in \mathscr{D}_{n}(\HH)$ then $F\bar{F} = -I_n$. \newline Taking the Study determinant of both sides, 
	\begin{align*}
		Sdet(E\bar{E}) &= Sdet(-I_n) \\
		Sdet(E)Sdet(\bar{E}) &= (-1)^n \\
		Sdet(E)\overline{Sdet(E)} &= (-1)^n \text{, by Theorem \ref{detbar}} \\
		|Sdet(E)|^2 &= (-1)^n
	\end{align*}
	Since $|Sdet(E)|^2 > 0$, $(-1)^n > 0$. Hence, $n$ must be even.
\end{proof}

\begin{remark}
	Theorem \ref{detbar} holds because the Study determinant is a complex determinant.
\end{remark}

	%\chapter{Summary and Recommendations}
	%In this chapter, we present a summary of what we've discussed in Chapters 2 and 3. We also present some recommendations for further study.

In Chapter 2, we introduced the notion of a complex vector space. We saw that the concept of a complex linear map and the concept of a complex determinant still hold. We also saw the concept of a \emph{complex structure} in a real vector space which allows us to mimic a complex vector space in a real vector space. We saw that the complex structure $J =
\begin{pmatrix}
 		0 & -1 \\
 		1 & 0
 \end{pmatrix}$ in $\R^2$
 corresponds to multiplication by $\ib$ in the complex vector space $\CC$.
 We saw that in order for a real linear map to correspond to a complex linear map, the said  real linear map has to commute with the complex structure defined in its corresponding real space. We also saw that we can only define complex structures for real vector spaces of even dimensions since the set of all $n \times n$ skew-coninvolutory matrices is empty when $n$ is odd.

 We also introduced the notion of a right quaternionic vector space. We deal with right quaternionic vector spaces because we saw that we cannot define a quaternionic linear map for left quaternionic vector spaces. We introduced the concept of a \emph{quaternionic structure} in a complex vector space which allows us to mimic a right quaternionic vector space in a complex vector space. We saw that a quaternionic structure $\Psi(v) = \overline{Jv}$ in $\CC^2$ where $J = 
\begin{pmatrix}
 		0 & -1 \\
 		1 & 0
 \end{pmatrix}$
 corresponds to right multiplication by $\jb$ in the right quaternionic vector space $\HH$. We saw that in order for a complex linear map correspond to a quaternionic linear map, the said complex linear map has to commute with the quaternionic structure defined in its corresponding complex space. 

In Chapter 3, we saw a problem in defining a determinant for quaternionic matrices. We saw that in order for a determinant to behave as expected, it must satisfy all three of Aslaksen's axioms and the consequence of which is that the determinant should map onto a commutative subset of $\HH$. We also saw that the Cayley determinant did not satisfy any of Aslaksen's axioms especially on the axiom concerning the determinant to be 0 for singular quaternionic matrices and that another approach was to be considered. The Study Determinant was one such approach that was shown in \cite{aslaksen} to satisfy all of the three axioms. 

In Chapter 3, we also introduced the matrix homomorphisms $\phi$ which allows us to represent $n \times n$ complex matrices as $2n\times 2n$ real matrices and $\psi$ which allows us to represent $n \times n$ quaternionic matrices as $2n\times 2n$ complex matrices. We also saw how $\phi$ and $\psi$ are used to define the Study Determinant. 

For further study, we recommend the following:

\begin{itemize}
	\item Expose the other quaternionic determinants, e.g., the Dieudonne determinant and Moore's determinant. There are other quaternionic determinants recommended in \cite{aslaksen}.
	\item Show that the set of all $n\times n$ skew-coninvolutory matrices is empty when $n$ is odd. Since the Study Determinant implies that $Sdet(-I_n) = (-1)^{2n} = 1^n$, we cannot use it to prove the latter. Use another quaternionic determinant instead.
	\item See if we can define a quaternionic structure in a complex vector space with odd dimensions. If we cannot, show that a quaternionic structure can only be defined on vector spaces with even dimensions.
\end{itemize}


	\appendix
	%\input{newfile}
	\nocite*

	\bibliographystyle{siam}
	\bibliography{bibliography}
\end{document}