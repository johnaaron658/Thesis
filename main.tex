\documentclass[12pt]{report}
\usepackage[utf8]{inputenc}
\usepackage[english]{babel}
\usepackage[mathscr]{euscript}
\usepackage{titlesec}
\usepackage{amsmath}
\usepackage{amsthm}
\usepackage{amssymb}
\usepackage{caption}
\usepackage{keyval}
\usepackage{subfig}
\usepackage{comment}
\usepackage{graphicx}
\usepackage{varwidth}
\usepackage{upcebumath}  
\usepackage{UPnotations}
\usepackage{hyperref}
\usepackage{rotating}
\usepackage{float}
\usepackage{gensymb}
\usepackage{mathabx}
%\usepackage{layout}
%\usepackage[hmargin={1.5in,1in},vmargin=1in,includehead,includefoot]{geometry}
\graphicspath{{images/}}

% THEOREM MACROS
%\theoremstyle{definition}
%\newtheorem{definition}{Definition}[section]
%\newtheorem{lemma}{Lemma}[section]
%\theoremstyle{theorem}
%\newtheorem{theorem}{Theorem}[section]
\newtheorem{prop}{Proposition}
\theoremstyle{definition}
\newtheorem{ex}{Example}[section]
%\theoremstyle{remark}
%\newtheorem*{remark}{Remark}
%\renewcommand\qedsymbol{$\blacksquare$}

% MATH SHORTCUTS
\makeatletter
\newcommand*\bigcdot{\mathpalette\bigcdot@{.5}}
\newcommand*\bigcdot@[2]{\mathbin{\vcenter{\hbox{\scalebox{#2}{$\m@th#1\bullet$}}}}}
\makeatother
\newcommand{\pvec}[1]{\vec{#1}\mkern2mu\vphantom{#1}}
\DeclareMathOperator{\R}{\mathbb{R}}
\DeclareMathOperator{\CC}{\mathbb{C}}
\DeclareMathOperator{\HH}{\mathbb{H}}
\newcommand{\Rx}[3]{{#1}_{#2}{(#3)}}
\newcommand{\Mr}[1]{\Rx{$M$}{#1}{\R}}
\newcommand{\Mc}[1]{\Rx{$M$}{#1}{\CC}}
\newcommand{\Mh}[1]{\Rx{$M$}{#1}{\HH}}
\newcommand{\Mf}[1]{\Rx{$M$}{#1}{$F$}}
\newcommand{\nbf}[1]{\textbf{#1}}
\newcommand{\ib}{\nbf{i}}
\newcommand{\jb}{\nbf{j}}
\newcommand{\kb}{\nbf{k}}
\newcommand{\ihat}{\hat{\ib}}
\newcommand{\jhat}{\hat{\jb}}
\newcommand{\khat}{\hat{\kb}}
\newcommand{\quat}[4]{#1 + #2\ib + #3\jb + #4\kb}
\newcommand{\ccdet}{det_{\CC}}
\newcommand{\rrdet}{det_{\R}}
\newcommand{\cdet}[1]{det_{\CC}(#1)}
\newcommand{\rdet}[1]{det_{\R}(#1)}
\newcommand{\genmatk}[4]{
	\begin{matrix}
	 #1 & \ldots & #2 \\
	 \vdots & \ddots & \vdots \\
	 #3 & \ldots & #4
	 \end{matrix}
}
\newcommand{\genmat}[1]{
	\genmatk{#1_{11}}{#1_{1n}}{#1_{n1}}{#1_{nn}}
}
\newcommand{\pgenmatk}[4]{
	\begin{pmatrix}
	 #1 & \ldots & #2 \\
	 \vdots & \ddots & \vdots \\
	 #3 & \ldots & #4
	 \end{pmatrix}
}
\newcommand{\pgenmat}[1]{
	\pgenmatk{#1_{11}}{#1_{1m}}{#1_{m1}}{#1_{mm}}
}
\newcommand{\vectC}[2]{\begin{pmatrix} #1 \\ #2 \end{pmatrix}}

% MATH ENVIRONMENT SETUP

%\setlength\arraycolsep{6pt}
%\renewcommand{\arraystretch}{0.75}
\newenvironment{proof}{\noindent \textit{Proof.}~}{\hfill$\square$}

% DOCUMENT SETUP
%\setlength{\parindent}{4em}
%\setlength{\parskip}{1em}
%\renewcommand{\baselinestretch}{2.0}
%\makeatletter
%\def\@makechapterhead#1{
%  {\parindent \z@ \raggedright \normalfont
%    \ifnum \c@secnumdepth >\m@ne
%        \huge\bfseries \@chapapp\space \thechapter
%        \par\nobreak
%        \vskip 20\p@
%    \fi
%    \interlinepenalty\@M
%    \Huge \bfseries #1\par\nobreak
%    \vskip 40\p@
%  }}
%\def\@makeschapterhead#1{
%  {\parindent \z@ \raggedright
%    \normalfont
%    \interlinepenalty\@M
%    \Huge \bfseries  #1\par\nobreak
%    \vskip 40\p@
%  }}
%\makeatother


% MAIN DOCUMENT
\begin{document}
	% DETAILS
	\title{On Quaternionic Linear Maps}

	\authorfirst{John Aaron Q.}
	\authorlast{Alcoseba}

	\degreemath{Bachelor of Science in Mathematics}
	\thesistype{Undergraduate Thesis}
	\thesisdocument{Thesis}
	\defensedate{May 29, 2017}
	\submitmonth{June}
	\submityear{2017}
	\adviser{Dr. Lorna S. Almocera}
	\reader{Dr. Natividad Estillore}

	\bothabsenttrue

	%\coadviser{co-Adviser, Ph.D.}

	\chairman{Jonnifer R. Sinogaya, Ph.D.} 
	\abs{This paper is an expository on quaternionic matrices and their implications on concepts in Linear Algebra such as the notion of vector spaces, linear maps, and determinants. We see that different definitions have been given for the latter. We choose to look into one such definition - the Study Determinant - which uses homomorphisms that represent $n\times n$ \emph{Complex Matrices} as $2n \times 2n$ real matrices, and representing $n\times n$ \emph{Quaternionic Matrices} as $2n \times 2n$ complex matrices. }
	\iffalse
	\acknowledge{
	Type your acknowledgements here. It is customary to acknowledge special assistance from
	extramural agencies. There is no obligation that assistance received from members of the dissertation or
	thesis committee be acknowledged. \medskip

	Acknowledgments should be couched in terms consistent with the
	scholarly nature of the work. Your name and the date should not appear on this page.
	}
	\fi

	\beforepreface
	\tableofcontents

	\newpage

	\acknowledgetrue
	\acknowledgepage

	\abstractpage
	%\listoftables
	%\listoffigures

	\afterpreface
	%main body
	\chapter{Introduction} 
	 A three-dimensional rotation can be expressed as a sequence of angles - each associated with a rotation about a principal axis. We call this set of angles \emph{Euler Angles} \cite{lerios} \cite{goldstein} \cite{strauch}. The \emph{principal axes} are initially the $x, y$ and $z$ axes. The axes change as the rotation takes place but, nonetheless, stay perpendicular to each other. 

\begin{figure}[H]
	\centering
	\subfloat[Initial Principal Axes]{\includegraphics[scale=0.345]{initial}\label{fig:init}}%
	\qquad
	\subfloat[Rotation by $\phi$ about the $z$-axis (blue)]{\includegraphics[scale=0.4]{phi_rotation}\label{fig:phirot}}%
	\\
	\subfloat[Rotation by $\theta$ about the new $y$-axis (red)]{\includegraphics[scale=0.4]{theta_rotation}\label{fig:thetarot}}%
	\qquad
	\subfloat[Rotation by $\psi$ about the new $z$-axis (blue)]{\includegraphics[scale=0.4]{psi_rotation}\label{fig:psirot}}
	\caption{\cite{strauch} \cite{goldstein} Rotation by Euler Angles}
	\label{eulerang}
\end{figure}

 As a demonstration, observe Figure \ref{eulerang}. Initially in \ref{fig:init}, we have the principal axes $y$ (red), $x$ (green) and $z$ (blue), each shown as arrows. We then rotate by an angle $\phi$ about the $z$-axis (blue) in \ref{fig:phirot}. Notice that as the rotation by the angle $\phi$ takes place, the $y$ (red) and $x$ (green) axes are displaced but remain perpendicular to $z$ and to each other. The displaced axes (each shown as a ball) now form a new set of principal axes from which we can then perform the next rotation in \ref{fig:thetarot}. Alternatively, we can visualize a rotation using a gyroscope as seen in Figure \ref{gyroscope}.

 \begin{figure}[H]
 	\centering
	\includegraphics[scale=0.5]{gyro}
 	\caption{Gyroscope - Rotation about $y$ (green/yellow ring), rotation about $z$ (blue ring), and rotation about $x$ (red ring).}
 	\label{gyroscope}
 \end{figure}

Computing for a rotation using Euler angles requires $4\times 4$ matrices like the ones shown in Figure \ref{4x4}. 

\begin{figure}[h]
	\begin{align*}
			\Rx{R}{x}{\alpha} &=
			\begin{pmatrix}
				1 & 0 & 0 & 0 \\
				0 & \cos{\alpha} & -\sin{\alpha} & 0 \\
				0 & \sin{\alpha} & \cos{\alpha} & 0 \\
				0 & 0 & 0 & 1
			\end{pmatrix}
			&
			\Rx{R}{y}{\beta} &=
			\begin{pmatrix}
				\cos{\beta} & 0 & \sin{\beta} & 0 \\
				0 & 1 & 0 & 0 \\
				-\sin{\beta} & 0 & \cos{\beta} & 0 \\
				0 & 0 & 0 & 1
			\end{pmatrix} \\
			(a) & \text{Rotation by $\alpha$ in the x-axis} 
			& 
			(b) & \text{Rotation by $\beta$ in the y-axis}	
	 \end{align*} 
		 \begin{align*}
			\Rx{R}{z}{\gamma} &=
			\begin{pmatrix}
				\cos{\gamma} & -\sin{\gamma} & 0 & 0 \\
				\sin{\gamma} & \cos{\gamma} &  0 & 0 \\
				0 & 0 & 1 & 0 \\
				0 & 0 & 0 & 1
		 \end{pmatrix} \\
		 (c) & \text{Rotation by $\gamma$ in the z-axis}	
		\end{align*}
		\caption{$4\times 4$ Rotation Matrices about the Principal Axes}
		\label{4x4}
\end{figure}

Computations involving the matrices in Figure \ref{4x4}, however, are a bit tedious and require more elementary arithmetic operations \cite{lerios}. It's also more difficult to determine the axis and angle of rotation \cite{lerios}. Furthermore, this method is susceptible to a problem in mechanics known as the \emph{Gimbal Lock} \cite{jia}.

The gimbal lock is a phenomenon that occurs when two successive rotations are about the same axis, i.e., if two of the rings in \ref{fig:lock} coincide - resulting in a loss of one degree of freedom for the object being rotated \cite{jia}. Losing a degree of freedom 

\begin{figure}
\centering
\subfloat[A rotation by $\phi$ then $\psi$ about the same axis.]{\includegraphics[scale=0.4]{gimbalLockrot}}%
\qquad
\subfloat[The blue and yellow/green rings coincide resulting in a loss of one degree of freedom.]{\includegraphics[scale=0.5]{gimbalLock}\label{fig:lock}}
\caption{Gimbal Lock}
\label{gimbal}
\end{figure}

%Ever since their discovery by William Rowan Hamilton in 1843, Quaternions have found extensive use in solving problems both in theoretical and applied mathematics - notably on the problem of 3D rotation.

Quaternions do not suffer from the gimbal lock. They are also found to be more compact - requiring less elementary arithmetic operations to perform rotation composition than rotation matrices \cite{lerios}. The axis and angle of rotation can also be easily deduced. Let $\vec{q}$ be the purely imaginary parts of the quaternion $q = a + b\ib + c\jb + d\kb $ ,i.e., $\vec{q} = b\ib + c\jb + d\kb$. It can be shown that $$\frac{\vec{q}}{\sqrt{b^2+c^2+d^2}} \text{ is the axis of rotation and }$$ $$\theta \text{ satisfying } \sin{\theta/2} = \sqrt{b^2+c^2+d^2} \text{ and } \cos{\theta/2} = a \text{ is the angle of rotation. \cite{lerios}}$$

Quaternions are used today in robotics, three-dimensional computer graphics, computer vision, crystallographic texture analysis, navigation, and molecular dynamics. 

Mathematicians have made advancements in developing the theory of Quaternions. Notably, as one of the central points of this topic, we look into the concept of a \emph{Quaternionic Matrix} and the implications it has on certain definitions that were already established in Linear Algebra. 

One such implication is the concept of a determinant in the context of quaternionic matrices. In linear algebra, we saw that we can extend the definition of the determinant to matrices with complex entries \cite{stamaria}. This is possible because the complex numbers are commutative under complex multiplication \cite{aslaksen}. 

Certain problems arise if we attempt to extend the classical definition to the quaternions because quaternions are not commutative under quaternion multiplication \cite{aslaksen}. In \cite{aslaksen}, Aslaksen revisits the properties we associate with determinants and gives three conditions called \emph{axioms} that should be satisfied in order for a definition of a determinant to be valid and useful:
\begin{enumerate}
	\item $det(A) = 0$ if and only if $A$ is singular.
	\item $det(AB) = det(A)det(B)$ for all quaternionic matrices $A$ and $B$.
	\item If $A'$ is obtained by adding a left-multiple of a row to another row or a right-multiple of a column to another column, then $det(A')=det(A)$.
\end{enumerate}

Over the years, several mathematicians have come up with different ways to define a determinant for quaternionic matrices - the Cayley determinant (by Arthur Cayley in 1845), the Study determinant (by Eduard Study in 1920), the Dieudonne determinant, and Moore's determinant. Aslaksen showed whether or not these different definitions satisfy the above conditions \cite{aslaksen}.

We will look at a particular problem that will require the concept of a determinant - that is, to determine whether or not the set of all $n \times n$ \emph{Skew-coninvolutory Quaternionic Matrices} (denoted by $\mathscr{D}_n(\HH)$) is empty when $n$ is odd. 

In \cite{stamaria}, Sta. Maria provided a simple proof to the fact that the set of all $n \times n$ skew-coninvolutory \emph{complex} matrices is empty when $n$ is odd. The method of proof involved using the determinant defined for complex matrices (which is not different from the classical determinant for matrices with real entries).

In this paper, we will discuss the theory behind quaternionic determinants - particularly, the Study determinant. We will then use the Study determinant to extend the result by Sta. Maria to the set of all skew-coninvolutory quaternionic matrices, i.e., we will show that the set of all $n \times n$ skew-coninvolutory quaternionic matrices is empty when $n$ is odd. 

\newpage
\section{List of Symbols}

\begin{itemize}
	\item $\Mr{n}$ - set of all $n\times n$ matrices with real entries.
	\item $\Mc{n}$ - set of all $n\times n$ matrices with complex entries.
	\item $\Mh{n}$ - set of all $n\times n$ matrices with quaternion entries.
	\item $\mathscr{D}_n(\C)$ - set of all $n\times n$ skew-coninvolutory matrices with complex entries.
	\item $\mathscr{D}_n(\HH)$ - set of all $n\times n$ skew-coninvolutory matrices with quaternion entries.
	\item $\cdet{ }$ - determinant of a matrix in $\Mc{n}$.
	\item $\rdet{ }$ - determinant of a matrix in $\Mr{n}$.
\end{itemize}

	\chapter{Preliminaries}
	\section{Complex Matrices}

\begin{definition}[Conjugate Matrix]
	A \emph{conjugate matrix} is a matrix $\bar{E}$ obtained from $E$ by taking the complex conjugate of every entry of $E$.
\end{definition}

\begin{definition}[Coninvolutory Matrix]
	A matrix is said to be \emph{coninvolutory} if $E\bar{E} = I_n$ for $E \in \Mc{n}$.
\end{definition}

\begin{remark}
	By manipulation, we obtain $E^{-1} = \bar{E}$. Hence, we may also say that a matrix whose inverse is its own conjugate matrix is a coninvolutory matrix. Furthermore, we see that coninvolutory matrices are the extension of complex numbers with modulus 1 \cite{stamaria}.
\end{remark}

\begin{definition}[Skew-Coninvolutory Matrix]
	A matrix is said to be \emph{skew-coninvolutory} if $E\bar{E} = -I_n$ for $E \in \Mc{n}$.
\end{definition}

\begin{remark}
	Again, we may say that a matrix whose inverse is the negative of its own conjugate matrix is a skew-coninvolutory matrix. This is analogous to the skew-symmetric matrices we've encountered in linear algebra.
\end{remark}

\begin{theorem} \label{detbar}
	For a matrix $E \in \Mc{n}$, $det(\bar{E}) = \overline{det(E)}$.
}
\end{theorem}

\begin{proof}
	We prove by mathematical induction. 
	\newline
	\newline \textbf{Base Case}: 
	For $E$ = 
	\begin{pmatrix}
		\bar{a} & \bar{b} \\
		\bar{c} & \bar{d}
	\end{pmatrix}, 
	$det(\bar{E})$ & =
	\begin{vmatrix}
		\bar{a} & \bar{b} \\
		\bar{c} & \bar{d}
	\end{vmatrix} $= \overline{ad} - \overline{bc} = \overline{ad-bc} = \overline{det(E)}$
	\newline
	\newline
	\textbf{Induction Hypothesis}:
	Suppose $det(\bar{E}) = \overline{det(E)}$ holds for $E \in \Mc{n}$.
	\newline
	Let $X \in \Mc{n+1}$. Then, $$det(\overline{X}) = \overline{\sum_{j=1}^{n+1} a_{ij}c_{ij}} = \sum_{j=1}^{n+1} \overline{a_{ij}}\text{ }\overline{c_{ij}}$$ is the $i^{th}$ row expansion of an $(n+1)\times (n+1)$ matrix where $\overline{c_{ij}}$ is the cofactor of $\overline{a_{ij}}$.
	\newline
	Note that $\overline{c_{ij}} = (-1)^{i+j}\overline{M_{ij}}$ where $\overline{M_{ij}}$ is the determinant of the $n\times n$ matrix obtained by deleting the $i^{th}$ row and the $j^{th}$ column of the original matrix.
	\newline
	By I.H., $\overline{M_{ij}}$ is the determinant of an $n\times n$ conjugate matrix. Thus, we see that we are computing for the determinant of an $(n+1)\times (n+1)$ conjugate matrix. 
\end{proof}

\subsection{Skew-Coninvolutory Complex Matrices}

We now show and prove a result concerning whether or not $\mathscr{D}_n(\C)$ is empty when $n$ is odd as seen in \cite{stamaria}.

\begin{theorem}
	$\mathscr{D}_{n}(\C)$ is empty when $n$ is odd.
\end{theorem}

\begin{proof}
	If $E \in \mathscr{D}_{n}(\C)$ then $E\bar{E} = -I_n$. \newline Taking the determinant of both sides, 
	\begin{align*}
		det(E\bar{E}) &= det(-I_n) \\
		det(E)det(\bar{E}) &= (-1)^n \\
		det(E)\overline{det(E)} &= (-1)^n \text{, by Theorem \ref{detbar}} \\
		|det(E)|^2 &= (-1)^n
	\end{align*}
	Since $|det(E)|^2 > 0$, $(-1)^n > 0$. Hence, $n$ must be even.
\end{proof}

\section{Quaternion Basics}

\subsection{Multiplication and Addition}

Recall in Chapter 1 - the four-dimensional algebra of quaternions is generated by the basis elements $\{1,\ib,\jb,\kb\}$ such that 
\begin{equation} \label{quat_eq}
\ib^2 = \jb^2 = \kb^2 = \ib \jb \kb = -1
\end{equation}

From the above equation, we can easily derive the following:
\begin{align*}
	\jb\kb &= \ib & \kb\jb &= -\ib \\
	\kb\ib &= \jb & \ib\kb &= -\jb \\
	\ib\jb &= \kb & \jb\kb &= -\kb
\end{align*}

Notice that the quaternions are not commutative under \emph{multiplication}. In general, for quaternions $q_1 = a_1 + b_1\ib + c_1\jb + d_1\kb$ and $q_2 = a_2 + b_2\ib + c_2\jb + d_2\kb$, 
\begin{align*}
	q_1q_2 &= (a_1a_2 - b_1b_2 - c_1c_2 - d_1d_2) + (a_1b_2 + b_1a_2 + c_1d_2 - d_1c_2)\ib \\
		   &+ (a_1c_2 - b_1d_2 + c_1a_2 + d_1b_2)\jb + (a_1d_2 + b_1c_2 - c_1b_2 + d_1a_2)\kb
\end{align*}

%insert vector representation

Quaternions are, however, commutative under \emph{addition} where $q_1+q_2 = (a_1+a_2) + (b_1+b_2)\ib + (c_1+c_2)\jb + (d_1+d_2)\kb$.

\subsection{Other Operations and Properties}

\begin{definition}[$\HH$-Conjugate]
	The $\HH$-Conjugate of a quaternion $q = \quat{a}{b}{c}{d}$ is $\bar{q} = a - b\ib - c\jb - d\kb$.
\end{definition}

\begin{remark}
	Notice that $q\bar{q} = (\quat{a}{b}{c}{d})(a - b\ib -c\jb-d\kb) = a^2+b^2+c^2+d^2$.
\end{remark}

\begin{definition}[$\HH$-Norm]
	The $\HH$-Norm of a quaternion $q = \quat{a}{b}{c}{d}$ is $|q| = \sqrt{q\bar{q}} = \sqrt{a^2+b^2+c^2+d^2}$
\end{definition}

\begin{definition}[Inverse]
	The inverse of a quaternion $q$ is $q^{-1}$ such that $q^{-1}q = qq^{-1} = 1$.
\end{definition}

\begin{theorem}
For $q, p, r \in \HH$,
	\begin{enumerate}
		\item $|q|^2 = q\bar{q}$.
		\item If $q\neq 0$, then $q^{-1} = \bar{q}/|q|^2$.
		\item $\overline{qp} = \bar{p}\bar{q}$.
		\item $(qp)^{-1} = p^{-1}q^{-1}$ provided that the inverses of $p$ and $q$ exist.
		\item $(qp)r = q(pr)$ that is, quaternion multiplication is associative.
	\end{enumerate}
\end{theorem}

%insert proof here.

\begin{remark}
	Notice that most of the properties we see in quaternions are merely extensions of the properties we see in complex numbers. 
\end{remark}

\subsection{Quaternionic Matrices}

Most of the definitions we've already mentioned for complex matrices can also be extended in the context of quaternionic matrices.

\begin{definition}[Conjugate Quaternionic Matrix]
	A \emph{conjugate quaternionic matrix} is a matrix $\bar{E}$ obtained from $E$ by taking the $\HH$-conjugate of every entry of $E$.
\end{definition}

\begin{definition}[Skew-Coninvolutory Quaternionic Matrix]
	A quaternionic matrix $E$ is said to be \emph{Skew-Coninvolutory} if $E\bar{E} = -I_n$.
\end{definition}

\section{Matrix Homomorphisms}

We look into functions that make it possible for us to represent complex numbers and quaternions as matrices. These functions are of extreme importance as they are used to define some of the quaternionic determinants we will encounter.

\subsection{Representing Complex Numbers as Real Matrices}

In abstract algebra, we saw that we can define an isomorphism from the field of complex numbers to the 2D-plane ($\R^2$) - a mapping $\phi : \C \rightarrow \R^2$ where a complex number $a+b\ib$ is mapped to a vector/point $(a,b)$ in the 2D-plane. Therefore, in order to represent complex numbers as real matrices, we have to find a way to view them as linear transformations over $\R^2$. 

Consider the complex function $f(z) = (a+b\ib)z$. We see that the images of $1$ and $\ib$ are $a+b\ib$ and $-b+a\ib$ respectively. 

\subsection{Homomorphisms from $\Mc{n}$ to $\Mr{2n}$}

\subsection{Representing Quaternions as Complex Matrices}

%exposition

\subsection{Homomorphisms from $\Mh{n}$ to $\Mc{2n}$}


\section{Quaternionic Determinants}

\subsection{Aslaksen's Axioms and the Cayley Determinant}

\subsection{The Study Determinant}

\subsection{The Dieudonne Determinant}

\subsection{Moore's Determinant}

%A quaternion $q = a + b\ib + c\jb + d\kb$ can be written as $q = (a+b\ib) + (c+d\ib)\jb = (a+b\ib) + (d-c\ib)\kb$ where $a+b\ib,c+d\ib,d-c\ib \in \C$. We can therefore view the set of quaternions as a two-dimensional algebra over \C \cite{stamaria}.


	\chapter{Results and Discussions}
	In this Chapter, we discuss the implications of quaternionic matrices on defining \emph{quaternionic determinants}. We then look into one Definition of a quaternionic determinant - the \emph{Study Determinant} - which uses matrix homomorphisms that allow us to represent complex matrices as real matrices and quaternionic matrices as complex matrices. 

\section{The Cayley Determinant and Aslaksen's Axioms} \label{cayley}

In 1845, 2 years after William Rowan Hamilton discovered quaternions, Arthur Cayley attempted to define the determinant of a quaternionic matrix using the usual formula (we denote the Cayley determinant by $Cdet$). Note that the following definition can be extended for $n\times n$ quaternionic matrices.

\begin{definition}[$2\times 2$ Cayley Determinant]
 \emph{\cite{aslaksen}} For a $2 \times 2$ quaternionic matrix \newline $A =  \begin{pmatrix} a & b \\ c & d \end{pmatrix}$, $Cdet(A) = ad - cb$ for $a,b,c,d \in \HH$. 
\end{definition}
 
 Notice that the order in which the entries are multiplied matters. We can see that for a $3\times 3$ quaternionic matrix 

 $B = 
\begin{pmatrix}
a & b & c \\
d & e & f \\
g & h & i
\end{pmatrix}$, $Cdet(B) = (aei+bfg+cdh)-(gec+hfa+idb)$.
 
 \begin{ex} \label{singbutnot}
 	\cite{aslaksen} Let $M = 
 	\begin{pmatrix}
 		\kb & \jb \\
 		\ib & 1
 	\end{pmatrix}$.
 	Then, $Cdet(M) = \kb-\ib\jb = \kb-\kb = 0$. Hence, we can say that by the Cayley determinant, $M$ is singular.
 \end{ex}
 \begin{ex}
 	\cite{aslaksen} Consider the transpose of the matrix $M$, $M^T = 
 	\begin{pmatrix} \label{invertbutnot}
 		\kb & \ib \\
 		\jb & 1
 	\end{pmatrix}$.
 	Then $Cdet(M^T) = \kb-(\jb\ib) = \kb - (-\kb) = 2\kb$. Hence, we can say that by the Cayley determinant, $M^T$ is invertible.
 \end{ex}
  Taking into account the fact that the quaternions are non-commutative, one might ask whether or not this determinant behaves the way we expect - Will it really determine whether or not a quaternionic matrix is singular or not? Will the properties of the determinant still hold? Will the determinant still be a map from $M_{n}(G) \rightarrow G$ (in this case, $G = \HH$) where $M_{n}(G)$ is the set of all $n\times n$ matrices over the elements of $G$? The last question comes from the fact that the determinant of a complex matrix yields a complex number.
}


\subsection{Brenner's Determinant Function and Aslaksen's Axioms}

We take a step back and revisit what it means for a mapping to be 	a determinant. J.L. Brenner \cite{brenner} and Helmer Aslaksen \cite{aslaksen} offer different approaches to this, however, we will see how we can both arrive at the same conclusions. The following definition is taken from \cite{brenner}.

\begin{definition}[Determinant Function] \label{detfn}
	\emph{\cite{brenner}} For a field $F$, a determinant over the matrices of $\Mf{n}$ is a function $det$ from $\Mf{n}$ into $F$ such that 
	\begin{equation}
	det(AB) = det(A)det(B) = det(B)det(A)
	\end{equation} 
	holds either
	\begin{enumerate}
		\item $\forall A, B \in \Mf{n}$ or
		\item $\forall$ invertible $A, B \in \Mf{n}$
	\end{enumerate}. 
\end{definition} 

Aslaksen, on the other hand, uses a more axiomatic approach in defining a determinant, presenting three determinant \emph{axioms} which a determinant definition must satisfy in order for it to behave the way we expect, i.e., it has the properties we associate with determinants. 
\begin{itemize}
	\item \textbf{Axiom 1.} $det(A) = 0$ if and only if $A$ is singular.
	\item \textbf{Axiom 2.} $det(AB) = det(A)det(B)$ for all quaternionic matrices $A$ and $B$.
	\item \textbf{Axiom 3.} If $A'$ is obtained by adding a left-multiple of a row to another row or a right-multiple of a column to another column, then $det(A')=det(A)$ (as we have already encountered in linear algebra, this operation can be described by an elementary matrix \cite{aslaksen}).
\end{itemize}

Notice that a determinant is essentially a function that:
\begin{enumerate}
 \item Maps to 0 if a matrix is singular
 \item Preserves multiplication and,
 \item Remains unchanged after applying the elementary operation of adding a left/right-multiple of one row/column to another row/column respectively.
\end{enumerate}

Also notice that Alaksen's second axiom is the first condition in Brenner's determinant function.

We can define the determinant simply as a function that constantly maps to 0 for all singular matrices and 1 for all non-singular matrices \cite{brenner}. This will satisfy the above axioms \cite{brenner} \cite{aslaksen}, however, we will mostly deal with determinants that are non-trivial (for instance the $\rrdet$, $\ccdet$, and $Cdet$). 

\begin{theorem} \label{nontrivdetfn}
\emph{\cite{brenner}} If $det$ is not constantly equal to 1 or 0 (i.e., $det$ is not a mapping $det: \Mf{n} \rightarrow F$ where $F$ is a field with two elements), then $det(B) = 0$ for all singular matrices. 
\end{theorem}

Theorem \ref{nontrivdetfn} not only shows how the determinant function in \ref{detfn} holds for the non-trivial case, it also shows that conditions (1) and (2) of Definition \ref{detfn} are essentially equivalent \cite{brenner}. This means that a determinant should map all singular matrices to 0. 

\begin{theorem} \label{comh}
	\emph{\cite{aslaksen}} Let $\Mh{n}$ be the set of all $n\times n$ quaternionic matrices. If $det$ satisfies all of Aslaksen's axioms, then $det(\Mh{n})$ is a commutative subset of $\HH$. 
\end{theorem}

Notice how Theorem \ref{comh} is already implied in Brenner's determinant function in which it is already deemed necessary for the images to commute. By Theorem \ref{comh}, we see that $det$ satisfying Aslaksen's axioms must only map to a commutative subset of $\HH$, which is the complex numbers. Therefore, $det$ cannot be a mapping onto $\HH$. Since $Cdet$ is onto $\HH$, by contrapositive of theorem \ref{comh}, $Cdet$ does not satisfy at least one of the axioms \cite{aslaksen}. 

\begin{ex}
\cite{aslaksen} As an illustration, we show that $Cdet$ doesn't satisfy Axiom 1. Recall examples \ref{singbutnot} and \ref{invertbutnot}. Notice that,
\begin{align*}
	&\text{\indent\indent\indent}M\begin{pmatrix}
	x \\
	y
	\end{pmatrix} = 
	\begin{pmatrix}
	0 \\
	0
	\end{pmatrix} \\ &\implies
	\begin{pmatrix}
	 		\kb & \jb \\
	 		\ib & 1
	\end{pmatrix}
	\begin{pmatrix}
	x \\
	y
	\end{pmatrix} = 
	\begin{pmatrix}
	0 \\
	0
	\end{pmatrix} \\ &\implies
	\kb x + \jb y = 0 \text{ and } \ib x + y = 0 \\ &\implies
	x = 0 \text{ and } y = 0. \\ &\implies
	M \text{ is invertible. This contradicts with the fact that $Cdet(M) = 0$.}
\end{align*}

Also notice that,

\begin{align*}
	&M^T\begin{pmatrix}
	-1 \\
	\jb
	\end{pmatrix} = 
	\begin{pmatrix}
	 		\kb & \ib \\
	 		\jb & 1
	\end{pmatrix}
	\begin{pmatrix}
	-1 \\
	\jb
	\end{pmatrix} = 
	\begin{pmatrix}
	0 \\
	0
	\end{pmatrix} \\ &\implies
	M^T \text{ is singular. This contradicts with the fact that $Cdet(M^T) = 2\kb$.}
\end{align*}
\end{ex}
 In \cite{aslaksen}, it is also shown that $Cdet$ doesn't satisfy Axioms 2 and 3.

\section{Matrix Homomorphisms}

Because the Cayley determinant fails to satisfy Aslaksen's axioms, the classical definition of the determinant cannot be extended to quaternionic matrices. It was not until 1920, that a new approach in defining a quaternionic determinant was presented in \cite{aslaksen}. The idea is to transform quaternionic matrices into complex matrices from which one could then just simply take the determinant \cite{aslaksen}. The method involves homomorphisms between quaternionic, complex, and real matrices. 

 In this section, we take a closer look into these homomorphisms - first discussing the motivation behind them and then the theory. 

\subsection{Representing Complex Numbers as Real Matrices}

Recall in Chapter 2 that $\CC$ forms a complex vector space. For us to represent a complex number $a+b\ib$ as a real matrix, we first have to express it as a linear map $f: \CC \rightarrow \CC$, i.e., a $1\times 1$ complex matrix $[a+b\ib]$. Thus, we have $f(z) = [a+b\ib]z$ which is identical to complex multiplication by $a+b\ib$. We see that the images of $1$ and $\ib$ under $f$ are $a+b\ib$ and $-b+a\ib$ respectively. We know from abstract algebra that we can define a bijection from the field of complex numbers to the 2D-plane ($\R^2$) - a mapping $\Theta : \CC \rightarrow \R^2$ where a complex number $a+b\ib$ is mapped to a vector/point $(a,b)$ in the 2D-plane. Under the function $\Theta$ (in which case $1$ is mapped to $(1,0)$ while $\ib$ is mapped to $(0,1)$), we seek a $2 \times 2$ real matrix that maps $(1,0)$ to $\Theta(a+b\ib) = (a,b)$ and $(0,1)$ to $\Theta(-b+a\ib) = (-b,a)$. 
\\
\noindent Let this matrix be $F =$ 
\begin{pmatrix} 
\alpha & \beta \\ 
\chi & \delta 
\end{pmatrix} where $\alpha, \beta, \chi, \text{ and } \delta \in \R$. Then, 
\begin{align*}
	\begin{pmatrix} 
		\alpha & \beta \\ 
		\chi & \delta 
	\end{pmatrix} 
	\begin{pmatrix} 
		1 \\ 0 
	\end{pmatrix} &= 
	\begin{pmatrix} a \\ b \end{pmatrix} \implies
	\begin{pmatrix}
		\alpha \\ \chi
	\end{pmatrix} =
	\begin{pmatrix} a \\ b \end{pmatrix} \implies 
	\alpha = a; \chi = b \text{ and} \\
	\begin{pmatrix} 
		\alpha & \beta \\ 
		\chi & \delta 
	\end{pmatrix} 
	\begin{pmatrix} 
		0 \\ 1 
	\end{pmatrix} &= 
	\begin{pmatrix} -b \\ a \end{pmatrix} \implies
	\begin{pmatrix}
		\beta \\ \delta
	\end{pmatrix} =
	\begin{pmatrix} -b \\ a \end{pmatrix} \implies
	\beta = -b; \delta = a
\end{align*}
\noindent Therefore, $F = \begin{pmatrix} \alpha & \beta \\ \chi & \delta \end{pmatrix} = \begin{pmatrix} \label{phismall} a & -b \\ b & a \end{pmatrix}$. 

\noindent\textit{Remark.} The matrix $F$ commutes with the complex structure in $\R^2$, $J = 
\begin{pmatrix}
 		0 & -1 \\
 		1 & 0
 \end{pmatrix}$, since, 
 \begin{align*}
	\begin{pmatrix}
	 	a & -b \\
	 	b & a
	 \end{pmatrix}
	\begin{pmatrix}
	 		0 & -1 \\
	 		1 & 0
	 \end{pmatrix} = 
	 \begin{pmatrix}
	 	-b & -a \\
	 	a & -b
	 \end{pmatrix} =
	 \begin{pmatrix}
	 		0 & -1 \\
	 		1 & 0
	 \end{pmatrix}
	 \begin{pmatrix}
	 	a & -b \\
	 	b & a
	 \end{pmatrix}
\end{align*}
Therefore, the matrix $F$ is a real linear map that represents the complex linear map $f$.

\begin{ex}
	Take the complex number $z = -3+2\ib$. Then its real matrix representation is
	$\begin{pmatrix}
		-3 & -2 \\
		2 & -3
	\end{pmatrix}$. 
	We can use this real matrix representation to multiply $z$ with another complex number, say $1+2\ib$. 
	$\begin{pmatrix}
		-3 & -2 \\
		2 & -3
	\end{pmatrix}
	\begin{pmatrix}
		1 \\ 2
	\end{pmatrix}
	 = 
	\begin{pmatrix}
		-7 \\
		-4
	\end{pmatrix}$
	which corresponds to the complex number obtained by multiplying $z(1+2\ib) = (1+2\ib)z = -7-4\ib$.
\end{ex}

It is important to note that $F$ in general, doesn't represent the complex number $a+b\ib$ itself but the linear map associated with multiplying a complex number by $a+b\ib$. In this case, $F$ can represent both left and right multiplication because complex numbers are commutative under multiplication. 

In representing the complex number $a+b\ib$ as a real matrix, we associate it with the real matrix $F$, i.e., we define a map $\phi : \CC \rightarrow \Mr{2}$ where $\Mr{2}$ is the set of all $2 \times 2$ real matrices, such that $a+b\ib \mapsto 
\begin{pmatrix}
	a & -b \\
	b & a
\end{pmatrix}$

In order for a real matrix representation to represent one and only one complex number $a+b\ib$, $\phi$ must be \emph{injective}. $\phi$ must also preserve the structure that complex multiplication gives to $\CC$, i.e., $\phi$ must also be a homomorphism. We've already shown in Chapter 2 Example \ref{ex:homomorph} that $\phi$ is indeed a homomorphism. In the next subsection, we show that $\phi$ is, in fact, an \emph{injective homomorphism} in the general case.

\subsection{Homomorphisms from $\Mc{n}$ to $\Mr{2n}$}

Let $\Mc{n}$ denote the set of all $n\times n$ complex matrices and $\Mr{2n}$ denote the set of all $2n \times 2n$ real matrices. When we represent complex matrices as real matrices, we are essentially representing complex linear maps as real linear maps. Recall in Chapter 2, that in order for a real linear map to represent a complex linear map, the said real map must commute with the complex structure defined in the corresponding real space $\R^{2n}$. 
We can define a matrix 
\begin{equation*} 
J = 
\begin{pmatrix} 
0 & -I_n \\ 
I_n & 0 
\end{pmatrix} 
\end{equation*}
We see that,
 \begin{align*}
 	J^2 = 
 	\begin{pmatrix}
 		0 & -I_n \\
 		I_n & 0
 	\end{pmatrix}
 	\begin{pmatrix}
 		0 & -I_n \\
 		I_n & 0
 	\end{pmatrix}
 	= 
 	\begin{pmatrix}
 		-I_n & 0 \\
 		0 & -I_n
 	\end{pmatrix}
 	= -I_{2n}
 \end{align*}
  Therefore, by Definition \ref{def:compstruct} $J$ gives a \emph{complex structure} in $\R^{2n}$. Let $\phi: \Mc{n} \rightarrow \Mr{2n}$ such that the image of a complex matrix under $\phi$ is its real matrix representation. Then $\phi(\Mc{n}) = \{P \in \Mr{2n} | JP = PJ\}$, i.e., the real matrix representations of complex matrices are those $2n \times 2n$ real matrices that commute with $J$ \cite{aslaksen}.   

The question now is which $2n \times 2n$ real matrices matrices commute with the complex structure in $\R^2$. Notice that every complex matrix can be represented as the sum of a real matrix and a purely imaginary matrix, i.e., for an $n\times n$ complex matrix $Z$, $Z = A + B\ib$ where $A,B \in \Mr{n}$ \cite{aslaksen}. We define a mapping 
\begin{equation} 
\label{eq:phimap}
	\phi(A+B\ib) = 
	\begin{pmatrix} 
	A & -B \\ B & A 
	\end{pmatrix}  
\end{equation} 

Notice that $F$ in \ref{phismall} is a special case of \ref{eq:phimap} where $n = 1$. If we can show that $\phi$ is an injective homomorphism (i.e., it preserves multiplication and is one-to-one), we can essentially represent any complex matrix as a $2n \times 2n$ real matrix. Also notice how the dimension of the real matrix is necessarily even. This is a direct consequence of Theorem \ref{dnc}.

\begin{theorem} \label{phimorph}
	Let $\phi: \Mc{n} \rightarrow \Mr{2n}$ such that $C+D\ib \mapsto $ \begin{pmatrix} C & -D \\ D & C \end{pmatrix} where $C+D\ib \in \Mc{n}$. Then $\phi$ is an injective homomorphism. 
\end{theorem}

\begin{proof}
	First, we show that $\phi$ is injective. Let $A+B\ib, C+D\ib \in \Mc{n}$. Then,
	\begin{align*}
		&\phi(A+B\ib) = \phi(C+D\ib) \\
		&\implies 
		\begin{pmatrix}
		A & -B \\ 
		B & A 
		\end{pmatrix} = 
		\begin{pmatrix}
		C & -D \\ 
		D & C 
		\end{pmatrix} \\
		&\implies A = C \text{ and } B = D \text{ by Matrix Equality} \\
		&\implies A+B\ib = C+D\ib \\
		&\implies \phi \text{ is injective.}
	\end{align*}
	
	We now show that $\phi$ is a homomorphism. Let $A+B\ib$, $C+D\ib \in \Mc{n}$. Then 
	\begin{align*}
		\phi[(A+B\ib)(C+D\ib)] &= \phi[(A+B\ib)C+(A+B\ib)D\ib] \text{ by Theorem \ref{distributive}} \\
		&= \phi[AC+BC\ib+AD\ib-BD] \\
		&= \phi[(AC-BD)+(BC+AD)\ib] \\
		&= 
		\begin{pmatrix} 
		AC-BD & -(BC+AD) \\ 
		BC+AD & AC-BD 
		\end{pmatrix}
	\end{align*}
	\begin{align*}
		\phi[(A+B\ib)]\phi[(C+D\ib)] &= 
		\begin{pmatrix}
		A & -B \\ 
		B & A 
		\end{pmatrix}
		\begin{pmatrix}
		C & -D \\ 
		D & C 
		\end{pmatrix} \\
		&=
		\begin{pmatrix} 
		AC-BD & -(BC+AD) \\ 
		BC+AD & AC-BD
		\end{pmatrix} \text{ by Theorem \ref{bigmatm}.} \\
		\text{Therefore, }\phi[(A+B\ib)(C+D\ib)] &= \phi[(A+B\ib)]\phi[(C+D\ib)] \text{ implying that $\phi$ is a homomorphism.}
	\end{align*}
Thus, $\phi$ is an injective homomorphism.
\end{proof}
\newline

 \begin{ex}
 	Take the complex matrix 

 	$Z = $
 	\begin{pmatrix}
 		1+2\ib & 3\ib \\
 		1 & -2+\ib
 	\end{pmatrix} $ = $
 	\begin{pmatrix}
 		1 & 0 \\
 		1 & -2
 	\end{pmatrix} $+$
 	\begin{pmatrix}
 		2 & 3 \\
 		0 & 1
 	\end{pmatrix} $\ib$.
 	Then, $\phi(Z) = $
 	\begin{pmatrix}
 		\begin{matrix}
 		1 & 0 \\
 		1 & -2
 		\end{matrix} & 
 		\begin{matrix}
 		-2 & -3 \\
 		1 & -1
 		\end{matrix} \\
 		\begin{matrix}
 		2 & 3 \\
 		0 & 1
 		\end{matrix} &
 		\begin{matrix}
 		1 & 0 \\
 		1 & -2
 		\end{matrix}
 	\end{pmatrix}
We know that $\phi(Z)$ is the only real matrix representation of $Z$ and that $\phi(Z)$ corresponds to the complex map $Z$ in the complex vector space $\CC^2$ since $\phi$ is an injective homomorphism.
 \end{ex}

\subsection{Representing Quaternions as Complex Matrices} \label{qrep}

Recall in Chapter 2 that we can view the quaternions as a 2-dimensional algebra over $\CC$ by having $\quat{a}{b}{c}{d} = (a+b\ib)+\jb(c-d\ib)$. In general, we can write any quaternion as $x+\jb y$ where $x, y \in \CC$. Because of this, we see that we can define a bijection $\Omega: \HH \rightarrow \CC^2$ where a quaternion $q = x+\jb y$ is mapped to $(x, y) \in \CC^2$. In representing a quaternion $q = x+\jb y$ as a complex matrix, we first have to express it as a linear map $s: \HH \rightarrow \HH$, i.e., a $1 \times 1$ quaternionic matrix $[x+\jb y]$. Thus, we have $s(q) = [x+\jb y]q$ which is identical to left quaternionic multiplication by $x+\jb y$. We see that,
\begin{align*}
	&s(1) = x+\jb y \\
	&s(\ib) = (x+\jb y)\ib = x\ib + \jb (y\ib) \\
	&s(\jb) = (x+\jb y)\jb = x\jb + \jb y\jb = \jb\bar{x}+\jb^2 \bar{y} = -\bar{y}+\jb\bar{x} \\
	&s(\kb) = (x+\jb y)\kb = x\kb + \jb y\kb = -x\jb\ib + \bar{y}\jb\kb = \bar{y}\ib - \jb(\bar{x}\ib) 
\end{align*}

\newcommand{\kmat}{\begin{pmatrix} \kappa & \lambda \\ \mu & \nu \end{pmatrix}}
\newcommand{\krmat}{\begin{pmatrix} \kappa_R & \lambda_R \\ \mu_R & \nu_R \end{pmatrix}}


Under the function $\Omega$, we see that the images of 1, $\ib, \jb$ and $\kb$ are $(1,0) , (\ib,0), (0,1)$ and $(0,-\ib)$ respectively and the images of $s(1), s(\ib), s(\jb)$ and $s(\kb)$ are $(x,y), (x\ib,y\ib), (-\bar{y},\bar{x})$ and $(\bar{y}\ib,-\bar{x}\ib)$. We seek a matrix in $\Mc{2}$ such that $(1,0) \mapsto (x,y), (\ib,0) \mapsto (x\ib,y\ib), (0,1) \mapsto (-\bar{y},\bar{x})$, and $(0,-\ib) \mapsto (\bar{y}\ib,-\bar{x}\ib)$.

 Let this matrix be $S = \kmat$ , where $\kappa, \lambda, \mu, $ and $\nu \in \CC$. Then, 

\begin{align*}
	&\kmat \vectC{1}{0} = \vectC{x}{y} \implies \vectC{\kappa}{\mu} = \vectC{x}{y} \implies \kappa = x \text{ and } \mu = y \\
	&\kmat \vectC{\ib}{0} = \vectC{x\ib}{y\ib} \implies \vectC{\kappa\ib}{\mu\ib} = \vectC{x\ib}{y\ib} \implies \kappa = x \text{ and } \mu = y \\
	&\kmat \vectC{0}{1} = \vectC{-\bar{y}}{\bar{x}} \implies \vectC{\lambda}{\nu} = \vectC{-\bar{y}}{\bar{x}} \implies \lambda = -\bar{y} \text{ and } \nu = \bar{x} \\
	&\kmat \vectC{0}{-\ib} = \vectC{\bar{y}\ib}{-\bar{x}\ib} \implies \vectC{-\lambda\ib}{-\nu\ib} = \vectC{\bar{y}\ib}{-\bar{x}\ib} \implies \lambda = -\bar{y} \text{ and } \nu = \bar{x}
\end{align*}

Hence, $S = \kmat = $
\begin{pmatrix}
	x & -\bar{y} \\
	y & \bar{x}
\end{pmatrix}.
\newline
\newline
\noindent\textit{Remark.} The matrix $S$ commutes with the quaternionic structure $\Psi$ in Chapter 2 Example \ref{ex:quatstruct} since for a complex vector $v = \vectC{x}{y} \in \CC^2$,
\begin{align*}
	&\begin{pmatrix}
		x & -\bar{y} \\
		y & \bar{x}
	\end{pmatrix}
	\overline{
		\begin{pmatrix}
		0 & -1 \\
		1 & 0
	\end{pmatrix}
	\vectC{x}{y}
	} = 
	\begin{pmatrix}
		x & -\bar{y} \\
		y & \bar{x}
	\end{pmatrix}
	\vectC{-\bar{y}}{\bar{x}} = 
	\vectC{-x\bar{y}-\bar{y}\bar{x}}{-y\bar{y}+\bar{x}^2} \text{ and } \\
	&\overline{
		\begin{pmatrix}
			0 & -1 \\
			1 & 0
		\end{pmatrix}
		\begin{pmatrix}
			x & -\bar{y} \\
			y & \bar{x}
		\end{pmatrix}
		\vectC{x}{y}
	} = 
	\overline{
		\begin{pmatrix}
			0 & -1 \\
			1 & 0
		\end{pmatrix}
		\vectC{x^2-\bar{y}y}{yx+\bar{x}y}
	} = \vectC{-x\bar{y}-\bar{y}\bar{x}}{-y\bar{y}+\bar{x}^2}
\end{align*}
Hence, the matrix $S$ is a complex linear map that represents the quaternionic linear map $s$.

\begin{ex}
	Take the quaternion $q = -2+\ib-5\jb+2\kb = -2+\ib + \jb(-5-2\ib)$. Then its complex matrix representation is 
	\begin{pmatrix}
		-2+\ib & 5-2\ib \\
		-5-2\ib & -2-\ib
	\end{pmatrix}.
	We can use this matrix to multiply $q$ with another quaternion, say $1+\ib+3\jb-4\kb = 1+\ib + \jb(3+4\ib)$. We have, 
	\begin{pmatrix}
		-2+\ib & 5-2\ib \\
		-5-2\ib & -2-\ib
	\end{pmatrix}
	$\vectC{1+\ib}{3+4\ib} = \vectC{20+13\ib}{-5-18\ib}$ which corresponds to the quaternion obtained by multiplying $q(1+\ib+3\jb-4\kb) = 20+13\ib-5\jb+18\kb = 20+13\ib+\jb(-5-18\ib)$ on the left.
\end{ex}
Notice that $S$ only represents left multiplication in the quaternions. One would probably ask whether or not we can have a complex matrix representation for right multiplication.
\newline
Consider the quaternionic function defined by multiplying a quaternion $q$ by some quaternion $x+\jb y$ on the right, $s_R(q) = q(x+\jb y)$ such that $y\neq 0$. We see that,
\begin{align*}
	s_R(1) &= x+\jb y \\
	s_R(\ib) &= \ib x - \jb\ib y \\
	s_R(\jb) &= -y + \jb x \\
	s_R(\kb) &= -\ib y -\jb\ib x 
\end{align*}

We use the same method in obtaining the complex matrix representation. We let $S_R = $ \krmat be the complex matrix representation of this function. 

\begin{align*}
	&\krmat \vectC{1}{0} = \vectC{x}{y} \implies \vectC{\kappa_R}{\mu_R} = \vectC{x}{y} \implies \kappa_R = x \text{ and } \mu_R = y \\
	&\krmat \vectC{\ib}{0} = \vectC{\ib x}{-\ib y} \implies \vectC{\kappa_R\ib}{\mu_R\ib} = \vectC{\ib x}{-\ib y} \implies \kappa_R = x \text{ and } \mu_R = -y \\ 
	&\krmat \vectC{0}{1} = \vectC{-y}{x} \implies \vectC{\lambda_R}{\nu_R} = \vectC{-y}{x} \implies \lambda_R = -y \text{ and } \nu_R = x \\
	&\krmat \vectC{0}{-\ib} = \vectC{-\ib y}{-\ib x} \implies \vectC{-\lambda_R\ib}{-\nu_R\ib} = \vectC{-\ib y}{-\ib x} \implies \lambda_R = y \text{ and } \nu_R = x 
\end{align*}

This implies that $\mu_R = \lambda_R = 0$. Hence, the complex matrix representation of right multiplication by a quaternion is 
\begin{pmatrix}
	x & 0 \\
	0 & x
\end{pmatrix}.
However, we see that, 
\begin{pmatrix}
	x & 0 \\
	0 & x
\end{pmatrix}
\vectC{1}{0} $ = $ \vectC{x}{0} which doesn't correspond to $s_R(1) &= x+\jb y$. Thus, we see that there is no matrix representation for right multiplication \cite{aslaksen}. This is because right scalar multiplication itself, if expressed as a mapping $L_R$ such that $L_R(v) = vq$ for $q \in \HH$, does not satisfy Property 2 of Definition \ref{linmapdef} between right vector spaces, i.e., $L_R(v)c = vrc \neq L_R(vc) = vcr$. In representing a quaternion $x+\jb y$ as a complex matrix, we associate it with the complex matrix $S$, i.e., we define a map $\psi: \HH \rightarrow \Mc{2}$ where $\Mc{2}$ is the set of all $2 \times 2$ complex matrices, such that $x+\jb y \mapsto
		\begin{pmatrix}
			x & -\bar{y} \\
			y & \bar{x}
		\end{pmatrix}$.

In order for a complex matrix representation to represent one and only one quaternion $x+\jb y$, $\psi$ must be injective. $\psi$ must also preserve the structure that right quaternion multiplication gives to $\HH$, i.e., $\psi$ must be a homomorphism. In the next subsection, we show that $\psi$ is indeed an injective homomorphism.

\subsection{Homomorphisms from $\Mh{n}$ to $\Mc{2n}$}

Let $\Mh{n}$ be the set of all $n\times n$ quaternionic matrices and $\Mc{2n}$ be the set of all $2n \times 2n$ complex matrices. When we represent quaternionic matrices as complex matrices, we are essentially representing quaternionic linear maps as complex linear maps. Recall in Chapter 2, that in order for a complex linear map to represent a quaternionic linear map, the said complex map must commute with the quaternionic structure defined in the corresponding complex space $\CC^{2n}$. 

For $v = \vectC{\chi}{\gamma} \in \CC^{2n}$ where $\chi = 
\begin{pmatrix}
	x_1 \\
	\vdots \\
	x_n
\end{pmatrix}$
 and $\gamma = 
 \begin{pmatrix}
	y_1 \\
	\vdots \\
	y_n
\end{pmatrix}$, we can define a mapping,
\begin{align*}
	\Psi(v) = 
	\overline{
	\begin{pmatrix}
		0 & -I_n \\
		I_n & 0
	\end{pmatrix}\vectC{\chi}{\gamma}}
	= \vectC{-\overline{\gamma}}{\overline{\chi}}
\end{align*}
We see that $\Psi$ satisfies Definition \ref{def:quatstruct}, i.e., $\Psi$ gives a \emph{quaternionic structure} in $\CC^{2n}$ since, 
\begin{align*}
	\Psi^2(v) = \Psi(\Psi(v)) = 
	\overline{
	\begin{pmatrix}
		0 & -I_n \\
		I_n & 0
	\end{pmatrix}\vectC{-\overline{\gamma}}{\overline{\chi}}}
	= \vectC{-\chi}{-\gamma} = -v.
\end{align*}
Let $\psi: \Mh{n} \rightarrow \Mc{2n}$ such that the image of a quaternionic under $\psi$ is its complex matrix representation. Then the complex matrix representations of quaternionic matrices are those $2n \times 2n$ complex matrices $N$ that commute with $\Psi$, i.e., 
\begin{equation} 
	N\overline{Jv} = \overline{JNv}.
	\label{eq:bigquatstruct}
\end{equation}
We can simplify Equation \ref{eq:bigquatstruct}. We have $N\overline{Jv} = \overline{JNv}$ if and only if $\overline{\overline{N}Jv} = \overline{JNv}$ which means that $\overline{N}J = JN$ or $NJ = J\overline{N}$. Notice that Theorem \ref{jx} in Chapter 2 (i.e. for $z \in \CC$, $z\jb = \jb\bar{z}$) is a special case of $NJ = J\overline{N}$ where $n = 1$. Hence, the complex matrices that commute with the quaternionic structure are those $2n \times 2n$ complex matrices $N$, such that $NJ = J\overline{N}$, i.e., $\psi(\Mh{n}) = \{N \in \Mc{2n} | NJ = J\overline{N}\}$. 

The question now is which $2n \times 2n$ complex matrices $N$ satisfy $NJ = J\overline{N}$. Notice that every quaternionic matrix can be represented as the sum $Q = X + \jb Y$ where $X, Y \in \Mc{n}$ \cite{aslaksen}. We define a mapping 
	\begin{equation} 
	\psi(X+\jb Y) = 
	\begin{pmatrix} 
	X & -\overline{Y} \\ 
	Y & \overline{X} 
	\end{pmatrix}  
	\label{eq:psimap}
\end{equation} 

Notice that the matrix $S$ is a special case of \ref{eq:psimap}. Again, if we can show that $\psi$ is an injective homomorphism, we can essentially represent any quaternionic matrix as a $2n \times 2n$ complex matrix.

\begin{theorem} \label{psimorph}
 	Let $\psi: \Mh{n} \rightarrow \Mc{2n}$ such that $X+\jb Y \mapsto $ 
 	\begin{pmatrix} 
 	X & -\overline{Y} \\ 
 	Y & \overline{X} 
 	\end{pmatrix} 
 	where $X+\jb Y \in \Mh{n}$. Then $\psi$ is an injective homomorphism. 
\end{theorem}

\begin{proof}
	We first show that $\psi$ is injective. Let $X+\jb Y, V+\jb W \in \Mh{n}$. Then,
	\begin{align*}
		&\psi(X+\jb Y) = \psi(V+\jb W) \\
		&\implies 
		\begin{pmatrix}
		X & -\overline{Y} \\ 
 		Y & \overline{X} 		
 		\end{pmatrix} = 
		\begin{pmatrix}
		V & -\overline{W} \\ 
 		W & \overline{V} 
		\end{pmatrix} \\
		&\implies X = V \text{ and } Y = W \text{ by Matrix Equality} \\
		&\implies X+\jb Y = V+\jb W \\
		&\implies \psi \text{ is injective.}
	\end{align*}
	We now show that $\psi$ is a homomorphism. Let $X+\jb Y$, $V+\jb W \in \Mh{n}$. Then 
	\begin{align*}
		\psi[(X+\jb Y)(V+\jb W)] &= \psi[X(V+\jb W)+\jb Y(V+\jb W)] \text{ by Theorem \ref{distributive}} \\
		&= \psi[XV+X\jb W + \jb YV+ \jb Y\jb W] \\
		&= \psi[XV+\jb \overline{X}W + \jb YV + \jb^2\overline{Y}W] \\
		&= \psi[(XV-\overline{Y}W)+\jb(\overline{X}W+YV)] \\
		&=
		\begin{pmatrix} 
		XV-\overline{Y}W & -\overline{(\overline{X}W+YV)} \\ 
		\overline{X}W+YV & \overline{XV-\overline{Y}W} 
		\end{pmatrix} \\
		&=
		\begin{pmatrix} 
		XV-\overline{Y}W & -X\overline{W}-\overline{Y}\overline{V} \\ 
		\overline{X}W+YV & \overline{X}\overline{V}-Y\overline{W} 
		\end{pmatrix}
		\end{align*}
		
		\begin{align*}
		\psi[(X+\jb Y)]\psi[(V+\jb W)] &= 
		\begin{pmatrix}
		X & -\overline{Y} \\ 
		Y & \overline{X} 
		\end{pmatrix}
		\begin{pmatrix}
		V & -\overline{W} \\ 
		W & \overline{V} 
		\end{pmatrix} \\
		&=
		\begin{pmatrix} 
		XV-\overline{Y}W & -X\overline{W}-\overline{Y}\overline{V} \\ 
		\overline{X}W+YV & \overline{X}\overline{V}-Y\overline{W} 
		\end{pmatrix} \text{ by Theorem \ref{bigmatm}.}
	\end{align*}
\end{proof}


\begin{ex}
	Take the quaternionic matrix 

	$Q = 
	\begin{pmatrix}
		1+2\ib-3\jb+\kb & 2\ib+5\kb \\
		1-\ib & 3+\jb+\kb
	\end{pmatrix}  = 
	\begin{pmatrix}
		1+2\ib & 2\ib \\
		1-\ib & 3
	\end{pmatrix}+\jb
	\begin{pmatrix}
		-3-\ib & -5\ib \\
		0 & 1-\ib
	\end{pmatrix}$.

	Hence, $\psi(Q) = 
	\begin{pmatrix}
		\begin{matrix}
			1+2\ib & 2\ib \\
			1-\ib & 3
		\end{matrix} &
		\begin{matrix}
			3-\ib & -5\ib \\
			0 & -1-\ib
		\end{matrix} \\
		\begin{matrix}
			-3-\ib & -5\ib \\
			0 & 1-\ib
		\end{matrix} &
		\begin{matrix}
			1-2\ib & -2\ib \\
			1+\ib & 3
		\end{matrix} 
	\end{pmatrix}$.
We know that $\psi(Q)$ is the only complex matrix representation of $Q$ and that $\psi(Q)$ corresponds to the quaternionic map $Q$ in the right quaternionic vector space $\HH^2$ since $\psi$ is an injective homomorphism.
\end{ex}
\newpage
\section{The Study Determinant}

The \emph{Study Determinant} uses the homomorphisms $\phi$ and $\psi$ to transform quaternionic matrices into complex or real matrices. We can then compute for the determinant of the complex or real matrix obtained by applying $\phi$ and $\psi$.

\begin{definition}[Study Determinant]
	\emph{\cite{aslaksen}} For $M \in \Mh{n}$, the \emph{Study Determinant} is defined by $Sdet(M) = \cdet{\psi(M)} = \sqrt{\rdet{\phi(\psi(M))}}$.
\end{definition}

Notice that we can compute for the Study determinant in two different ways: 
\begin{enumerate}
	\item Simply getting the complex matrix representation of the quaternionic matrix and proceeding to take its complex determinant or;
	\item Getting the real matrix representation of the quaternionic matrix by composing $\phi$ and $\psi$, and then taking the square root of the real determinant.
\end{enumerate}

In the following example, we take the Study determinant of the quaternionic matrix in Example \ref{singbutnot} for which the Cayley determinant failed to identify as non-singular. 

\begin{ex}
	Let $M = $ 
	\begin{pmatrix}
 		\kb & \jb \\
 		\ib & 1
 	\end{pmatrix} $ = $
 	\begin{pmatrix}
 		0 & 0 \\
 		\ib & 1
 	\end{pmatrix} $+\jb$
 	\begin{pmatrix}
 		-\ib & 1 \\
 		0 & 0
 	\end{pmatrix}.
 	Then,

 	\begin{enumerate}
 		\item $Sdet(M) = \ccdet(\psi(M)) = \ccdet
		\begin{pmatrix}
			\begin{matrix}
				0 & 0 \\
				\ib & 1
			\end{matrix} &
			\begin{matrix}
				-\ib & -1 \\
				0 & 0
			\end{matrix} \\
			\begin{matrix}
 				-\ib & 1 \\
 				0 & 0
 			\end{matrix} &
 			\begin{matrix}
				0 & 0 \\
				-\ib & 1
			\end{matrix}
 		\end{pmatrix}  = 4$. 
 		\item We see that $\psi(M) = 
 		\begin{pmatrix}
 			0 & 0 & 0 & -1 \\
 			0 & 1 & 0 & 0 \\
 			0 & 1 & 0 & 0 \\
 			0 & 0 & 0 & 1
 		\end{pmatrix} +
 		\begin{pmatrix}
 			0 & 0 & -1 & 0 \\
 			1 & 0 & 0 & 0 \\
 			-1 & 0 & 0 & 0 \\
 			0 & 0 & -1 & 0
 		\end{pmatrix} \ib$.

 		$\sqrt{\rrdet(\phi(\psi(M)))} =$
 		$\begin{pmatrix}
 		\rrdet
 		\begin{pmatrix}
 			0 & 0 & 0  & -1 &  0 & 0 & 1 &  0 \\
 			0 & 1 & 0  &  0 & -1 & 0 & 0 &  0 \\
 			0 & 1 & 0  &  0 &  1 & 0 & 0 &  0 \\
 			0 & 0 & 0  &  1 &  0 & 0 & 1 &  0 \\
 			0 & 0 & -1 &  0 &  0 & 0 & 0 & -1 \\
 			1 & 0 & 0  &  0 &  0 & 1 & 0 &  0 \\
 		   -1 & 0 & 0  &  0 &  0 & 1 & 0 &  0 \\
 		    0 & 0 & -1 &  0 &  0 & 0 & 0 &  1 \\
 		\end{pmatrix}
 		\end{pmatrix}^{1/2} = \sqrt{16} = 4$
 	\end{enumerate}


\end{ex}

Hence, we can see that by the Study determinant, the matrix $M$ is invertible which it really is as shown in section \ref{cayley}. 

The Study Determinant holds for \textbf{Axiom 2} since $\psi$ and $\phi$ are homomorphisms \cite{aslaksen}.

As for \textbf{Axiom 1}, notice that for an $n\times n$ quaternionic matrix $M$, $\ccdet(\psi(M))$ is a complex determinant. When $\ccdet(\psi(M)) \neq 0 $ , the complex matrix $\psi(M)$ is invertible. We need to see if the inverse of $\psi(M)$ is also a complex matrix representation of a quaternionic linear map, i.e., $\psi(M)^{-1} \in \psi(\Mh{n}) = \{N \in \Mc{2n} | NJ = J\overline{N}\}$ \cite{aslaksen}. Since $\psi(M) \in \psi(\Mh{n})$, $\psi(M)J = J\overline{\psi(M)}$. By getting the inverse of both sides, we get 
\begin{align*}
	(\psi(M)J)^{-1} &= (J\overline{\psi(M)})^{-1}\\
	J^{-1}\psi(M)^{-1} &= \overline{\psi(M)}^{-1}J^{-1}\\
	J\psi(M)^{-1} &= \overline{\psi(M)^{-1}}J
\end{align*}
Hence, the inverse is in $\psi(\Mh{n})$. This implies that \textbf{Axiom 1} holds for the Study Determinant.

We can represent the elementary row operation described in \textbf{Axiom 3} using an $n \times n$ matrix \\$B_{ij}(b) = 
		\begin{pmatrix}
			1 & 0 & \cdots & 0 \\
			0 & 1 & \cdots & 0 \\
			0 & b & \cdots & 0 \\
			\vdots & \vdots & \ddots & \vdots \\
			0 & 0 & \cdots & 1 
		\end{pmatrix}
		=
		\begin{pmatrix}
			1 & 0 & 0 & \cdots & 0 \\
			0 & 1 & 0 & \cdots & 0 \\
			0 & 0 & 1 & \cdots & 0 \\
			\vdots & \vdots & \vdots & \ddots & \vdots \\
			0 & 0 & 0 & \cdots & 1
		\end{pmatrix} + b
		\begin{pmatrix}
			0 & 0 & 0 & \cdots & 0 \\
			0 & 0 & 0 & \cdots & 0 \\
			0 & 1 & 0 & \cdots & 0 \\
			\vdots & \vdots & \vdots & \ddots & \vdots \\
			0 & 0 & 0 & \cdots & 0
		\end{pmatrix} = I_n + be_{ij}$ \\where $e_{ij}$ is the $n\times n$ matrix with 1 on the $ij^{th}$ entry (for $i \neq j$) and 0 elsewhere \cite{aslaksen}. Notice that $e_{ij}e_{ij}=0$ since $i \neq j$.

		Multiplying by $B_{ij}(b)$ on the left adds the left scalar multiple by $b$ of the $j^{th}$ row to the $i^{th}$ row and multiplying by $B_{ij}(b)$ on the right adds the right scalar multiple by $b$ of the $i^{th}$ column to the $j^{th}$ column \cite{aslaksen}.

		Since by \textbf{Axiom 2}, $Sdet(B_{ij}(b)M) = Sdet(B_{ij}(b))Sdet(M)$, we need to show that $Sdet(B_{ij}(b)) = 1$ in order to show that $Sdet(M)$ doesn't change after applying $B_{ij}(b)$. Before we proceed, we present Theorem \ref{theorem:detmatgen} which allows us to compute for the determinant of $2m\times 2m$ matrices.

\begin{theorem} \label{theorem:detmatgen}
	\emph{\cite{aslaksen}} If $A_{11}, A_{12}$ and $A_{22}$ are $m \times m$ matrices that are mutually commutative, then,
	\begin{equation*}
		det
		\begin{pmatrix}
			A_{11} & A_{12} \\
			A_{21} & A_{22}
		\end{pmatrix}
		= det(A_{11}A_{22} - A_{12}A_{21})
	\end{equation*}
\end{theorem}

		If $b = x + \jb y$, then $B_{ij}(b) = I_n + xe_{ij} + \jb ye_{ij}$, hence,

		$\psi(B_{ij}(b)) = 
		\begin{pmatrix}
			I_n + xe_{ij} & -\bar{y}e_{ij} \\
			ye_{ij} & I_n+\bar{x}e_{ij}
		\end{pmatrix}$ and,
		\begin{align*}
			\ccdet(\psi(B_{ij}(b))) &= \ccdet[(I_n+xe_{ij})(I_n+\bar{x}e_{ij})+y\bar{y}e_{ij}e_{ij}] \\ &\text{ by Theorem \ref{theorem:detmatgen}.} \\
			&=\ccdet[(I_n+xe_{ij})(I_n+\bar{x}e_{ij})] \\
			&\text{since $e_{ij}e_{ij} = 0$.} \\
			&=\ccdet(I_n+xe_{ij})(\overline{I_n+xe_{ij}}) \\
			&=\ccdet(I_n+xe_{ij})\overline{(\ccdet(I_n+xe_{ij}))} \\
			&\text{by Theorem \ref{detbar}.} \\
			&=|\ccdet(I_n+xe_{ij})|^2 \\
			&=|\ccdet(I_n)|^2 \text{ since $B_{ij}(x)$ is a triangular matrix.} \\
			&= 1
		\end{align*}
		Hence, the Study Determinant satisfies \textbf{Axiom 3}.


\iffalse
\section{Skew-Coninvolutory Quaternionic Matrices of Odd Dimensions}
In Chapter 2, we saw in Theorem \ref{dnc} that there are no \emph{skew-coninvolutory complex matrices} of odd dimensions. In the real case, Theorem \ref{dnc} greatly influenced the dimensions of real vector spaces in which a \emph{complex structure} can be defined. We will show that Theorem \ref{dnc} also holds for quaternionic matrices, i.e., the set of all $n \times n$ skew-coninvolutory quaternionic matrices is empty when $n$ is odd.

\begin{theorem}[Main Result]
 Let $\mathscr{D}_n(\HH)$ denote the set of all skew-coninvolutory quaternionic matrices. Then $\mathscr{D}_n(\HH)$ is empty when $n$ is odd.
\end{theorem}
\begin{proof}
	If $F \in \mathscr{D}_{n}(\HH)$ then $F\bar{F} = -I_n$ by Definition \ref{skewquatmat} 
	\newline 
	We see that we can take the Study Determinant of both sides since $F$ is a quaternionic matrix, 
	$$Sdet(F\bar{F}) &= Sdet(-I_n)$$
	Notice that,
	\begin{align*}
		Sdet(-I_n) = \ccdet(\psi(-I_n)) = 
		\begin{pmatrix}
			\genmatk{-1}{0}{0}{-1}
		\end{pmatrix}
	\end{align*}
	\begin{align*}
		Sdet(F)Sdet(\bar{F}) &= (-1)^n \\
		Sdet(F)\overline{Sdet(F)} &= (-1)^n \text{, by Theorem \ref{detbar}} \\
		|Sdet(F)|^2 &= (-1)^n
	\end{align*}
	Since $|Sdet(F)|^2 > 0$, $(-1)^n > 0$. Hence, $n$ must be even.
\end{proof}

	Theorem \ref{detbar} applies because the Study determinant can be viewed as a complex determinant.

\begin{ex}
	Again consider the $1 \times 1$ case where, $\jb\bar{\jb} = -\jb^2 = 1 \neq -1$. Whereas in the $2 \times 2$ case, consider $E = $ 
	\begin{pmatrix}
		0 & \jb \\
		-\jb & 0
	\end{pmatrix}. Then, $E\overline{E} = $
	 \begin{pmatrix}
		0 & \jb \\
		-\jb & 0
	\end{pmatrix}
	\begin{pmatrix}
		0 & -\jb \\
		\jb & 0
	\end{pmatrix} $ = $
	\begin{pmatrix}
		-1 &  0 \\
		 0 & -1
	\end{pmatrix}.
\end{ex}
\fi

	\chapter{Summary and Recommendations}
	In this chapter, we present a summary of what we've discussed in Chapters 2 and 3. We also present some recommendations for further study.

In Chapter 2, we introduced the notion of a complex vector space. We saw that the concept of a complex linear map and the concept of a complex determinant still hold. We also saw the concept of a \emph{complex structure} in a real vector space which allows us to mimic a complex vector space in a real vector space. We saw that the complex structure $J =
\begin{pmatrix}
 		0 & -1 \\
 		1 & 0
 \end{pmatrix}$ in $\R^2$
 corresponds to multiplication by $\ib$ in the complex vector space $\CC$.
 We saw that in order for a real linear map to correspond to a complex linear map, the said  real linear map has to commute with the complex structure defined in its corresponding real space. We also saw that we can only define complex structures for real vector spaces of even dimensions since the set of all $n \times n$ skew-coninvolutory matrices is empty when $n$ is odd.

 We also introduced the notion of a right quaternionic vector space. We deal with right quaternionic vector spaces because we saw that we cannot define a quaternionic linear map for left quaternionic vector spaces. We introduced the concept of a \emph{quaternionic structure} in a complex vector space which allows us to mimic a right quaternionic vector space in a complex vector space. We saw that a quaternionic structure $\Psi(v) = \overline{Jv}$ in $\CC^2$ where $J = 
\begin{pmatrix}
 		0 & -1 \\
 		1 & 0
 \end{pmatrix}$
 corresponds to right multiplication by $\jb$ in the right quaternionic vector space $\HH$. We saw that in order for a complex linear map correspond to a quaternionic linear map, the said complex linear map has to commute with the quaternionic structure defined in its corresponding complex space. 

In Chapter 3, we saw a problem in defining a determinant for quaternionic matrices. We saw that in order for a determinant to behave as expected, it must satisfy all three of Aslaksen's axioms and the consequence of which is that the determinant should map onto a commutative subset of $\HH$. We also saw that the Cayley determinant did not satisfy any of Aslaksen's axioms especially on the axiom concerning the determinant to be 0 for singular quaternionic matrices and that another approach was to be considered. The Study Determinant was one such approach that was shown in \cite{aslaksen} to satisfy all of the three axioms. 

In Chapter 3, we also introduced the matrix homomorphisms $\phi$ which allows us to represent $n \times n$ complex matrices as $2n\times 2n$ real matrices and $\psi$ which allows us to represent $n \times n$ quaternionic matrices as $2n\times 2n$ complex matrices. We also saw how $\phi$ and $\psi$ are used to define the Study Determinant. 

For further study, we recommend the following:

\begin{itemize}
	\item Expose the other quaternionic determinants, e.g., the Dieudonne determinant and Moore's determinant. Look into the notion of a \emph{quasideterminant}, i.e., a determinant for matrices over non-commutative division rings.
	\item Show that the set of all $n\times n$ skew-coninvolutory matrices is empty when $n$ is odd. Since the Study Determinant implies that $Sdet(-I_n) = (-1)^{2n} = 1^n$, we cannot determine the nature of $n$ and thus, we cannot use the Study Determinant to prove the latter using a similar proof outline provided in Theorem \ref{dnc}.
	\item See if we can define a quaternionic structure in a complex vector space with odd dimensions. If we cannot, show that a quaternionic structure can only be defined on vector spaces with even dimensions.
\end{itemize}


	\appendix
	%\input{newfile}
	\nocite*

	\bibliographystyle{siam}
	\bibliography{bibliography}
\end{document}